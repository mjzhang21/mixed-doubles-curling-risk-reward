% LTeX: language=en-US
\documentclass[titlepage, 12pt]{article}
\usepackage{amsmath}
\usepackage[margin=1in]{geometry}
\usepackage{graphicx}
\usepackage[utf8]{inputenc}
\usepackage{natbib}
\usepackage[colorlinks, citecolor=blue, urlcolor=blue, linkcolor=blue]{hyperref}

\usepackage{xcolor}
\usepackage{caption}
\usepackage{subcaption}  

\graphicspath{{img/}{writing/img/}}

\newcommand{\jy}[1]{{\textcolor{violet}{JY:\@ #1}}}
\newcommand{\mz}[1]{{\textcolor{teal}{MZ:\@ #1}}}
% Uncomment this line before submission
% \renewcommand{\jy}[1]{}
%\ renewcommand{\mz}[1]{}

\title{Strategic Adaptation to the Power Play in Mixed Doubles Curling}

% \date{[15 January 2026]}



\begin{document}
\maketitle

\begin{abstract}
  The power play in mixed doubles curling was introduced to open
  the sheet and encourage offensive scoring, yet its strategic
  effects remain largely undocumented. Using stone-level
  coordinate and shot assessment data from three World Mixed
  Doubles Curling Championships and one Winter Olympics, this
  study examines strategic behavior associated with power play
  usage and response. End-level intent is inferred from
  shot-level task information by distinguishing early and late
  phases defined by the Modified Free Guard Zone, combining
  rule-based classification with unsupervised clustering to
  characterize offensive and defensive patterns. Across more
  than 5,000 team-ends, power play ends show a systematic shift
  toward defensive behavior in the late phase relative to
  non-power-play ends, particularly in score-protective
  contexts, and this shift is not driven solely by hammer-team
  behavior. Analysis of the non-hammer team’s first shot against
  the power play identifies three dominant counterplay patterns
  based on spatial stone configurations. These patterns differ
  in execution quality and cumulative score dynamics but do not
  yield statistically significant differences in end-level
  scoring outcomes.
\end{abstract}

\section{Introduction}

\jy{Paragraphs are fragmented. I like longer paragraphs, each 130-160 words,
  with strong topic sentences. A reader could sketch the topic sentences only
  to get a quick idea about the paper.}

\jy{Three questions to address in introduction and in this order:
  why important; what've been done; what's new.}
\mz{created three paragraphs and added more references}

Mixed doubles curling presents a unique strategic environment with recent rule
changes—such as the modified free guard zone, pre-placed stones, and the power
play—that alter tactics and increase the importance of adaptive planning
\citep{WorldCurlingRules2025}. Developed at the International Olympic
Committee's request for a faster, more variable format, it lacks extensive
data-driven strategies. The power play, a key innovation, can be used once per
match when a team has the hammer, and it opens the sheet’s center by
repositioning the pre-placed stones. This shift changes incentives and
necessitates clear guidelines for calling, executing, and responding to the
power play.

Strategy, curling physics, and robotics are some of the major research areas in
curling \citep{zacharias2024examination, maeno2016assignments, gwon2020path}.
Research specifically in curling strategy use methods including decision
analysis, logistic regression and deep reinforcement learning
\citep{willoughby2005analysis, clement2012analysis, han2022game}. For example
\citet{willoughby2005analysis} and \citet{clement2012analysis} address the
long-debated question on whether blanking or scoring one is the better choice
in the seventh end. Furthermore, \citet{park2013curling} confirms the advantage
of possessing the hammer with statistical evidence. However, these studies do
not specifically account for the unique dynamics introduced by the recent rule
changes in mixed doubles curling, particularly the power play. Given that mixed
doubles and the power play introduce different strategic considerations, it is
essential to extend research efforts to this format to develop data-driven
guidelines

Using data from three World Championships and one Winter Olympics, this paper
offers the first analysis of strategic behavior surrounding the power play in
professional mixed doubles curling. We show that late-phase behavior shifts
toward defense in power play ends, with the largest and most persistent shift
coming from the non-calling team. We quantify timing effects, provide
team-level descriptive strategy profiles, and validate the behavioral proxy
with stone-removal outcomes. We then focus on non-hammer team behavior by
analyzing common counter power play shot types and show that different openings
do not significantly alter end-level scoring outcomes, even though they produce
meaningful differences in execution quality and cumulative score dynamics.
Despite teams systematically varying shot choice with game context, the data
provide no evidence that any single counter-strategy is objectively dominant
when opening an end against the power play.

\section{Data}

We use Curlit's mixed doubles tracking data, which records shot-by-shot stone
positions and analyst annotations across three World Mixed Doubles
Championships and one Winter Olympics. The dataset contains 26,730 shot rows
and includes unique identifiers for competition, game, end, team, player, and
shot order, so that each match can be reconstructed chronologically. In
addition to x--y stone coordinates after every throw, Curlit provides
analyst-graded fields for intended task and execution quality, which we use to
classify behavior.

This is the first publicly available mixed doubles dataset that combines
stone-level geometry with end-level context and shot assessment. These features
allow us to link strategic choices to both sheet state and game state. End
results are recorded for both teams, which enables reconstruction of score
progression and pre-end score differential.

At the shot level, the distribution is draw-heavy, consistent with the MFGZ and
an opportunistic or probing early phase. Figure~\ref{fig:thrown_positions_all}
summarizes the spatial density of thrown stone locations across all shots.
Power play usage is observed in a minority of ends and varies by end number,
and task-group shares shift between power play and non-power-play ends
(Figure~\ref{fig:data_pp_summary}).

\begin{figure}[htbp]
  \centering
  \includegraphics[width=0.6\textwidth]{img/thrown_positions_all.pdf}
  \caption{Stone position density across all shots, illustrating the regions
    of the sheet most frequently occupied after a throw.}
  \label{fig:thrown_positions_all}
\end{figure}

\begin{figure}[htbp]
  \centering
  \begin{subfigure}[b]{0.47\textwidth}
    \centering
    \includegraphics[width=\linewidth, height=0.23\textheight,
      keepaspectratio]{img/powerplay_end_frequency.pdf}
    \caption{Power play usage by end number.}
    \label{fig:powerplay_end_frequency}
  \end{subfigure}
  \hfill
  \begin{subfigure}[b]{0.47\textwidth}
    \centering
    \includegraphics[width=\linewidth, height=0.23\textheight,
      keepaspectratio]{img/task_by_powerplay.pdf}
    \caption{Task-group shares by power play status.}
    \label{fig:task_by_powerplay}
  \end{subfigure}
  \caption{Power play usage context and task composition.}
  \label{fig:data_pp_summary}
\end{figure}

\section{Power Play and Late-Phase Shot Selection}
\subsection{Data}
\subsection{Methodology}

Conceptually, we define behavior as offensive, defensive, or probing
(opportunistic). These behaviors can appear tactically within a single end
(shot-to-shot choices) and strategically across ends (an intended end goal or
game plan). Because probing behavior is inherently reactive and not uniquely
encoded in the task labels, we treat neutral or ambiguous shots as the closest
empirical proxy for probing in this analysis.

Our behavioral analysis is conducted at the team-end level, which creates a
paired structure within each end and allows a direct comparison between the
power play caller and the opponent. We focus on the late phase of the end (the
final seven deliveries) because early shots are heavily constrained by the
Modified Free Guard Zone and the pre-placed stones. Late shots therefore
provide the clearest window into strategic choice rather than rule-driven
necessity.

Behavior is operationalized through shot selection, using the task labels
assigned by analysts. We group tasks into offensive placements (draws, guards,
raises, freezes, and hit-and-rolls), defensive removals (peels, take-outs,
clears, doubles, and promotions), and a neutral category for throughs or
no-shot entries. This choice is deliberate: the task coding is the only
consistently available, interpretable indicator of intent at the shot level.
Position-based intent would require model-driven inference and introduce a
second layer of assumptions, so we treat shot selection as the observable
behavioral signal and reserve positional data for validation.

Power play exposure is assigned at both the end and team-end levels. An end is
marked as a power play end if either team uses its power play, and within such
ends we identify the caller and the non-caller. This yields a three-category
comparison group: non-power-play ends, power-play callers, and power-play
non-callers. Each team-end is also annotated with hammer status, pre-end score
differential, end number, team identity, and match identity to control for
state and repeated context.

For each team-end we compute late-phase shares of offensive, defensive, and
neutral shots and define the behavior index as defensive share minus offensive
share, where larger values indicate a more defensive profile. We also split the
late phase into early-late (shots 16--18) and late-late (shots 19--22) to test
timing mechanisms. Because hammer teams throw four late shots and non- hammer
teams throw three, shares are in discrete increments; using shares standardizes
exposure across teams and ends.

The primary specification is fixed-effects OLS with clustering at the match
level. Fixed effects for team and match absorb persistent strategy and match-
specific conditions, while end fixed effects control for systematic end-of-
game dynamics. This specification favors interpretability of the power play
coefficients in share points and is conservative with respect to within-match
correlation. Diagnostics include residual plots, Q--Q plots, and influence
measures. As a validation, we test whether higher defensive shares correspond
to more opponent stones removed during late shots, using the off-sheet sentinel
to identify removals. Limitations include the discrete nature of the outcome,
potential selection into power play usage, and reliance on task coding as a
proxy for intent.

Formally, the main model is:
\[
  \begin{aligned}
    \text{behavior\_index}_{i e} =\; & \beta_0 + \beta_1 \text{PP\_caller}_{i e}
    + \beta_2 \text{PP\_noncaller}_{i e}                                                     \\
                                     & + \beta_3 \text{has\_hammer}_{i e}
    + \beta_4 \text{pre\_end\_score\_diff}_{i e}                                             \\
                                     & + \alpha_e + \gamma_i + \delta_m + \varepsilon_{i e},
  \end{aligned}
\]
where $i$ indexes team-ends, $e$ indexes ends, and $m$ indexes matches. The
omitted reference category is non-power-play ends, and standard errors are
clustered at the match level. We estimate the same specification for the lateA
and lateB timing splits.

\subsection{Results}

Across the late-phase sample (5,274 team-ends), non-power-play ends are
predominantly offensive. The mean defensive share is 0.295 while the offensive
share is 0.690, yielding a mean behavior index of $-0.395$. Power-play ends
shift toward defense for both teams. For callers, the defensive share rises to
0.426 and the offensive share falls to 0.566 (behavior index $-0.141$). For
non-callers, the defensive share is 0.455 and the offensive share is 0.540
(behavior index $-0.085$). These descriptive contrasts show a clear defensive
shift in power-play ends, with the larger shift occurring for the opponent.

Power play usage is concentrated in non-tied score states: it is used in 0.113
of tied ends versus 0.265 when trailing or leading. Within power-play ends, the
defensive shift is strongest in protective contexts. For example, when the
caller is leading, the caller's defensive share rises to 0.658 and the
non-caller's rises to 0.587. Timing splits show that the caller's adjustment
appears early, while the non-caller remains defensive throughout: in lateA,
defensive shares are 0.266 (non-PP), 0.417 (caller), and 0.457 (non-caller),
and in lateB they are 0.314, 0.434, and 0.455, respectively.
Figure~\ref{fig:pp_def_share_by_score} summarizes the score-state pattern by
plotting mean late-phase defensive share across trailing, tied, and leading
contexts for each power-play group.

\begin{figure}[htbp]
  \centering
  \includegraphics[width=0.8\textwidth]{img/pp_def_share_by_score.pdf}
  \caption{Mean late-phase defensive share by score state, with separate lines
    for non-power-play ends, power-play callers, and power-play non-callers.}
  \label{fig:pp_def_share_by_score}
\end{figure}

The fixed-effects regression corroborates these descriptive patterns after
controlling for hammer, score differential, end, team, and match
(Table~\ref{tab:pp_regression}). In the main model, the PP caller coefficient
is 0.068 ($p<0.05$) while the PP non-caller coefficient is 0.449 ($p<0.01$),
indicating a much larger opponent response. Timing models show a caller effect
in lateA (0.128, $p<0.01$) but not in lateB (0.009, $p=0.82$), while the
non-caller remains strongly defensive in both lateA (0.516) and lateB (0.414).

Positional validation strengthens the behavioral interpretation: the Spearman
correlation between defensive share and opponent stones removed late is 0.804.
Taken together, the results indicate that power-play ends induce a defensive
shift for both teams, with a substantially larger and more persistent response
from the non-caller.

\begin{table}[t]
  \centering
  \caption{Power play effects on late-phase behavior index}
  \label{tab:pp_regression}
  \begin{tabular}{lccc}
    \hline
    & Main & Mid & Close \\
    \hline
    PP caller & 0.068** & 0.128*** & 0.009 \\
    & (0.030) & (0.038) & (0.038) \\
    PP non-caller & 0.449*** & 0.516*** & 0.414*** \\
    & (0.035) & (0.046) & (0.043) \\
    \hline
    N & 5,274 & 5,274 & 5,274 \\
    R$^2$ & 0.232 & 0.221 & 0.161 \\
    End FE & Yes & Yes & Yes \\
    Team FE & Yes & Yes & Yes \\
    Match FE & Yes & Yes & Yes \\
    \hline
  \end{tabular}
  \begin{flushleft}
  \footnotesize Notes: Cluster-robust standard errors at the match level in
  parentheses. $^{*}$ $p<0.10$, $^{**}$ $p<0.05$, $^{***}$ $p<0.01$.
  \end{flushleft}
\end{table}


To address team strategy profiles, Table~\ref{tab:team_profiles} reports
descriptive late-phase behavior by team. These profiles summarize each team’s
late-phase defensive share, offensive share, and behavior index, alongside
power-play calling rate and hammer frequency. They are intended as descriptive
benchmarks rather than causal estimates.

\begin{table}[t]
  \centering
  \caption{Team strategy profiles (late-phase behavior and power play usage)}
  \label{tab:team_profiles}
  \begin{tabular}{lrrrrrrr}
    \hline
    NOC & TeamID & N & Def. & Off. & Index & PP call & Hammer \\
    \hline
    JPN & 16 & 226 & 0.367 & 0.621 & -0.253 & 0.124 & 0.504 \\
    HUN & 36 & 71 & 0.367 & 0.622 & -0.255 & 0.127 & 0.535 \\
    EST & 37 & 263 & 0.360 & 0.624 & -0.264 & 0.122 & 0.510 \\
    ITA & 24 & 311 & 0.356 & 0.629 & -0.274 & 0.090 & 0.434 \\
    FRA & 13 & 80 & 0.343 & 0.626 & -0.283 & 0.125 & 0.600 \\
    USA & 20 & 311 & 0.350 & 0.641 & -0.290 & 0.103 & 0.437 \\
    SUI & 18 & 302 & 0.343 & 0.637 & -0.293 & 0.099 & 0.493 \\
    SWE & 19 & 306 & 0.349 & 0.643 & -0.294 & 0.105 & 0.493 \\
    CAN & 10 & 313 & 0.342 & 0.644 & -0.301 & 0.089 & 0.489 \\
    SCO & 14 & 247 & 0.340 & 0.646 & -0.306 & 0.109 & 0.506 \\
    ESP & 34 & 203 & 0.340 & 0.651 & -0.311 & 0.123 & 0.542 \\
    TUR & 51 & 218 & 0.333 & 0.649 & -0.315 & 0.124 & 0.528 \\
    AUS & 46 & 300 & 0.338 & 0.654 & -0.317 & 0.113 & 0.487 \\
    CZE & 22 & 296 & 0.332 & 0.649 & -0.317 & 0.118 & 0.493 \\
    CHN & 43 & 220 & 0.324 & 0.659 & -0.335 & 0.123 & 0.509 \\
    KOR & 44 & 218 & 0.321 & 0.666 & -0.345 & 0.119 & 0.491 \\
    DEN & 11 & 219 & 0.296 & 0.689 & -0.393 & 0.123 & 0.530 \\
    NZL & 30 & 132 & 0.294 & 0.691 & -0.397 & 0.129 & 0.553 \\
    NOR & 17 & 339 & 0.293 & 0.694 & -0.401 & 0.115 & 0.469 \\
    GER & 15 & 208 & 0.294 & 0.695 & -0.401 & 0.120 & 0.543 \\
    ENG & 23 & 60 & 0.297 & 0.703 & -0.406 & 0.133 & 0.583 \\
    NED & 25 & 207 & 0.272 & 0.711 & -0.439 & 0.121 & 0.536 \\
    FIN & 12 & 66 & 0.254 & 0.739 & -0.485 & 0.136 & 0.485 \\
    GBR & 27 & 86 & 0.243 & 0.739 & -0.496 & 0.116 & 0.512 \\
    AUT & 28 & 72 & 0.233 & 0.756 & -0.523 & 0.111 & 0.486 \\
    \hline
  \end{tabular}
  \begin{flushleft}
  \footnotesize Notes: Def. and Off. are late-phase defensive and offensive shot shares. Index = Def. - Off. PP call is the share of team-ends in which the team called power play. Hammer is the share of team-ends with hammer.
  \end{flushleft}
\end{table}


\subsection{Discussion}

Analysts describe the power play as a mechanism to clear the center of the
sheet by moving the pre-placed stones to the side, which should make offense
easier. At the same time, practitioner intuition (e.g., CurlTech and Rocks
Across the Pond) suggests that teams often invoke the power play late and in a
defensive posture. Our results reconcile these views. The late-phase behavior
shift is decisively toward defense for both teams, and the non-caller’s shift
is substantially larger and more persistent than the caller’s. This confirms
the analysts’ intuition that power play does not simply amplify offense and, in
practice, often corresponds to conservative late-phase shot selection. It also
disconfirms the common expectation that the hammer team becomes more offensive
in power play ends; the hammer-only shift is modest compared to the opponent’s
response.

The state dependence of usage supports a strategic interpretation. Power play
is used much less when tied and more when teams are either trailing or leading,
precisely the contexts in which variance management matters most. The cleared
sheet lowers the cost of removal play and raises the risk of large swings, so
both teams appear to prioritize control once takeouts are available. The
caller’s early defensive adjustment is consistent with stabilizing the end,
while the non-caller’s sustained defensive profile suggests a deliberate
attempt to neutralize the caller’s positional leverage in open ice.

These patterns translate into actionable guidance. For the hammer team, power
play should be treated as a situational control tool rather than a guaranteed
offensive boost. If the objective is to protect a lead, early defensive
sequencing appears effective; if trailing, teams should anticipate a strong
defensive counter and build scoring chances early before removal play
dominates. For the non-hammer team, the evidence supports proactive disruption
and removal as the default response, with a clear plan for when to transition
from probing offense to conservative control.

Methodologically, we interpret behavior through shot selection because it is
the only directly observed indicator of intent; positional validation supports
this proxy, but the estimates are behavioral associations rather than causal
effects. This framing sets up the clustering-based discussion that follows by
providing an end-level interpretation of power play behavior before we examine
the non-hammer team’s first-shot response patterns.

\section{First-shot Non-hammer Team Powerplay Adaptation}
\subsection{Data}

Before analysis, the data was preprocessed by combining datasets, removing
irrelevant rows and swapping columns. After the coordinates were standardized,
the stone-level data were combined with end-level information and filtered to
include only the first shot of the powerplay in each right-formation end.
Furthermore, the preplaced stone columns were swapped so that the first
coordinates referred to the non-hammer team's guard and the second pair
referred to the hammer-team stone near the button. Finally, rows containing
missing data were dropped, which effectively removed shots that eliminated
stones prior to entering the Modified Free Guard Zone.

\subsection{Methodology}

To explore strategic patterns in first-shot powerplays, stone positions were
analyzed using dimensionality reduction techniques (PCA, t-SNE, UMAP) and
clustered with HDBSCAN. This combination allowed complex six-dimensional data,
including x and y coordinates of preplaced and first stones, to be visualized
in two dimensions while revealing clusters corresponding to common tactical
placements. Cluster quality was assessed using silhouette scores, membership
probabilities, and coverage metrics. Scatter plots on the curling sheet
illustrated typical stone positions for each cluster, providing an view of
different board configurations. Dimensionality reduction, clustering, and
spatial visualization were used to reveal tactical patterns, while also
capturing quantitative details of shot placement.

Performance differences across clusters were evaluated using Welch’s ANOVA with
bootstrap confidence intervals to account for skewed distributions and unequal
variances, ensuring robust comparisons of group means. When significant
differences were detected, Games–Howell post hoc tests identified which
clusters contributed to observed variation. To examine international variation
in strategic choices, first non-hammer shots in left and right formation
powerplays were visualized using histograms and bar charts, capturing both shot
distributions and frequencies. Only teams with more than twenty powerplay ends
were included to ensure meaningful comparisons. Similar shot types were
grouped, and the top five countries for each type were plotted to reveal
national tendencies in responding to powerplays. These visualizations and
groupings allowed clear comparisons of shot selection patterns across clusters
and countries.

\subsection{Results}

The HDBSCAN clustering produced four distinct clusters, capturing board
patterns in the first-shot powerplay ends. The coverage rate, DBCV score, and
silhouette score (0.604) indicate a reasonably strong clustering structure
despite some noise. Clusters 0–3 had mean membership probabilities of 0.773,
0.753, 0.891, and 0.982, respectively, indicating that Cluster 3 contained the
most tightly grouped points, while Clusters 0 and 1 had weaker assignments.
Visualizations on PCA, t-SNE and UMAP confirmed well separated clusters with
minimal overlap (Figure~\ref{fig:fs_tSNE}).Overall, the clustering identifies
meaningful groups that can be analyzed for strategic differences.

\begin{figure}[t]
  \centering
  \includegraphics[width=1.0\textwidth]{img/fs_t-SNE.pdf}
  \caption{“t-SNE visualization of the first-shot power-play left-formation
    dataset in two dimensions.
    Each point represents a sample, colored by cluster.
    Clusters indicate similarity in the feature space.”}
  \label{fig:fs_tSNE}
\end{figure}

To further characterize these clusters, we examined the spatial distribution of
stone placements and shot types. Plotting the stone coordinates on a curling
sheet three distinct board configurations, shown in Figure~\ref{fig:C_all}.
Some stones interacted with preplaced guards, while other leave them untouched
and prioritize placement near the house or in the front. Cluster 3 represented
a central draw pattern but was excluded from further analysis due to small
sample size ($n$ = 7).We next examined the distribution of shot types within
each cluster. Cluster 0 exhibited a roughly equal proportion of raise/tapbacks
and wick/soft-peeling shots, Cluster 1 consisted entirely of draws, and Cluster
2 had an equal mix of front and guard placements. Together, these spatial and
shot-type patterns highlight the primary variations in early powerplay
strategies across clusters.

\begin{figure}[t]
  \centering

  \begin{subfigure}[b]{0.32\textwidth}
    \centering
    \includegraphics[width=\textwidth]{img/C0.pdf}
    \caption{Cluster 0: Tick shot}
    \label{fig:C0}
  \end{subfigure}
  \hfill
  \begin{subfigure}[b]{0.32\textwidth}
    \centering
    \includegraphics[width=\textwidth]{img/C1.pdf}
    \caption{Cluster 1: Draw}
    \label{fig:C1}
  \end{subfigure}
  \hfill
  \begin{subfigure}[b]{0.32\textwidth}
    \centering
    \includegraphics[width=\textwidth]{img/C2.pdf}
    \caption{Cluster 2: Front/Guard}
    \label{fig:C2}
  \end{subfigure}

  \caption{Coordinate visualizations after the first shot in the power-play
    left formation. Each panel highlights a different cluster, which can be
    identified as shot types: tick (Panel A), draw around the guard (Panel B),
    and front/guard (Panel C).}
  \label{fig:C_all}
\end{figure}

Finally, we assessed whether cluster membership was associated with differences
in game outcomes. Cumulative score difference differed significantly across
clusters (Welch’s ANOVA, $p=0.017$; Table~\ref{tab:cluster_test}), with the
highest mean observed in C1 and the lowest in C0. Post-hoc Games--Howell tests
indicated that this effect was driven by a significant difference between C0
and C1, whereas C2 did not differ significantly from either cluster. Points
also varied across clusters (Welch’s ANOVA, $p=0.005$), with a significant
difference observed between C1 and C2 only. In contrast, hammer and non-hammer
scores did not differ significantly across clusters. Overall, these results
indicate that cluster structure was associated with variation in overall
pre-end score differences and execution score, but not with hammer-specific
performance.

\jy{n is not $n$}
\mz{changed}

\documentclass{article}

\usepackage[utf8]{inputenc} % only needed for pdflatex
\usepackage{booktabs, tabularx, threeparttable, graphicx, caption}

\pagestyle{empty}
\begin{document}

% Optional: reduce column padding and font size for compact table
\setlength{\tabcolsep}{4pt}
\small

\begin{table}[h!]
\centering

\begin{threeparttable}
\begin{tabularx}{\textwidth}{>{\raggedright\arraybackslash}X c c c c}
\toprule
Variables & C0 (n=35) & C1 (n=75) & C3 (n=67) & p-value \\
\midrule
Cumulative score difference\tnote{*}    & 3.51 (2.06--4.97) & 6.00 (5.07--6.93) & 4.84 (3.73--5.94) & 0.015 \\
Points\tnote{\textdagger}               & 4.00 (3.00--4.00) & 3.50 (3.00--4.00) & 4.00 (3.00--4.00) & 0.003 \\
Hammer score\tnote{\textdagger}         & 1.00 (0.50--2.00) & 1.00 (1.00--2.00) & 1.00 (0.50--2.00) & 0.958 \\
Non-hammer score\tnote{\textdagger}     & 0.00 (0.00--0.50) & 0.00 (0.00--0.00) & 0.00 (0.00--0.00) & 0.904 \\
\bottomrule
\end{tabularx}

\begin{tablenotes}[flushleft]
\small
\item[*] Welch’s one-way ANOVA
\item[\textdagger] Kruskal-Wallis H test
\item Pairwise comparisons: Cumulative score difference (Games–Howell): C0 vs C1 $p=0.014$, C0 vs C3 $p=0.317$, C1 vs C3 $p=0.245$.
\item Pairwise comparisons: Points (Dunn test): C0 vs C1 $p=0.129$, C0 vs C3 $p=1.000$, C1 vs C3 $p=0.002$.
\end{tablenotes}
\end{threeparttable}
\end{table}

\end{document}


\subsection{Discussion}
Analysts have observed different responses for countering the power play
\citep{rocks2021mixed}, although there have been no studies to identify exactly
which shots teams hit and their effect on performance and context metrics.
Analysts have observed different responses for countering the power play
\citep{rocks2021mixed}, although there have been no studies that identify
exactly which shots teams hit and their effect on performance and context
metrics. By clustering board positions after the non-hammer team’s first shot,
three distinct patterns emerge in mixed doubles championships: tick, guard, and
draw around the guard. Understanding these patterns is critical because the
first shot of the end dictates the flow and subseuqent shot selection. Also,
identifying international responses is crucial for developing team specific
strategies in anticipation of a certain match.

Welch ANOVA was used to compare differences between clusters, revealing
significant differences in cumulative score difference and points in non-missed
shots, while hammer and non-hammer scores did not differ significantly. Post
hoc Games–Howell tests show that the cumulative score differences are driven by
contrasts between the tick and draw-around-the-guard clusters. All shots are
utilized when the hammer team has the score advantage; however, tick shots are
more common when the score difference is small, whereas draws are favored when
the hammer team leads by a margin. These patterns suggest teams may throw less
risky when trailing modestly. Nevertheless, since hammer and non-hammer scores
do not vary across clusters, shot choice may not strongly influence end-level
performance.This suggests non-hammer teams select the starting player based on
overall skill in executing one of the three observed options. Limitations
include small powerplay sample size and skewed features, particularly in
non-hammer score.

When considering all power play ends, countries differ in their first-shot
strategies. The bar chart of task-group proportions by nation highlights these
variations (Figure~\ref{fig:noc_pptask}). While draws and tick shots are common
across teams, some national tendencies stand out, reflecting different
approaches to countering the power play. These patterns align with analysts’
expectations about strategic variation and demonstrate how teams adapt their
play style in response to the power play. Further analysis of subsequent shots
and early shot sequences could provide additional insight into how teams build
on initial positioning throughout the end.

\begin{figure}[t]
  \centering
  \includegraphics[width=\textwidth]{img/noc_pptask.pdf}
  \caption{“Proportions of grouped tasks hit by country. Data includes all
    power play ends and is limited to the first stone. Only the top five teams
    with more than 20 power play ends are shown.”}
  \label{fig:noc_pptask}
\end{figure}

\jy{Could be brief, because this is not exactly an academic paper but a
  report for reference for the coaching team.}
\mz{Removed section, will address briefly in discussion section}

\section{Conclusion}

We find that the power play is associated with a consistent late-phase shift
toward defensive shot selection for both teams, with the non-caller showing the
larger and more persistent adjustment. These shifts are concentrated in
non-tied score states and align with a control-oriented interpretation of the
open sheet. The behavioral proxy based on task selection is supported by
positional validation, suggesting it captures meaningful intent under the data
constraints.

At the shot level, clustering of the non-hammer team’s first response reveals
three dominant counterplay patterns. These include the tick shot, draw around
the guard and a center guard. We conclude that while shot execution and score
difference when executed varies, the first stone shot selection does not impact
non-hammer or hammer-team end scores. While teams do show a clear strategy for
their first stone, it is likely other factors drive powerplay end-level
success.

\jy{Do you have team strategy profiles?}
\mz{Added team offensive defense metrics along with teams with distinct first shot choice
}
\bibliographystyle{apalike}
\clearpage
\bibliography{references.bib}

\end{document}
