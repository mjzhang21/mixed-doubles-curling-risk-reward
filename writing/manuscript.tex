% LTeX: language=en-US
\documentclass[titlepage, 12pt]{article}
\usepackage{amsmath}
\usepackage[margin=1in]{geometry}
\usepackage{graphicx}
\usepackage[utf8]{inputenc}
\usepackage{natbib}
\usepackage[colorlinks, citecolor=blue, urlcolor=blue, linkcolor=blue]{hyperref}

\usepackage{xcolor}
\usepackage{caption}
\usepackage{subcaption}  

\graphicspath{{img/}{writing/img/}}

\newcommand{\jy}[1]{{\textcolor{violet}{JY:\@ #1}}}
\newcommand{\mz}[1]{{\textcolor{teal}{MZ:\@ #1}}}
% Uncomment this line before submission
% \renewcommand{\jy}[1]{}
%\ renewcommand{\mz}[1]{}

\title{Strategic Adaptation to the Power Play in Mixed Doubles Curling}

% \date{[15 January 2026]}



\begin{document}
\maketitle

\begin{abstract}
  The power play in mixed doubles curling was introduced to open
  the sheet and encourage offensive scoring, yet its strategic
  effects remain largely undocumented. Using stone-level
  coordinate and shot assessment data from three World Mixed
  Doubles Curling Championships and one Winter Olympics, this
  study examines strategic behavior associated with power play
  usage and response. End-level intent is inferred from
  shot-level task information by distinguishing early and late
  phases defined by the Modified Free Guard Zone, combining
  rule-based classification with unsupervised clustering to
  characterize offensive and defensive patterns. Across more
  than 5,000 team-ends, power play ends show a systematic shift
  toward defensive behavior in the late phase relative to
  non-power-play ends, particularly in score-protective
  contexts, and this shift is not driven solely by hammer-team
  behavior. Analysis of the non-hammer team’s first shot against
  the power play identifies three dominant counterplay patterns
  based on spatial stone configurations. These patterns differ
  in execution quality and cumulative score dynamics but do not
  yield statistically significant differences in end-level
  scoring outcomes.
\end{abstract}

% Note: % is comment character, \ backslash is the escape character (needed for
% $, %, etc.)

% New paragraphs will be formed automatically similar to quarto, just leave a
% blank line in between lines. Inline math can be called by wrapping with $, for
% example $ math equation$.

\section{Introduction}

% briefly outline what the purpose of the data challenge is exactly
% -> there isn o data anlysis done on mixed doubles curling
% all mixed doubles curling strategy and intuition relies on
% strategy formulated from the tradtional game.
% this is why we have the data challenge, is so that we can develop
% strategy independent of the 4 person game.

\jy{Paragraphs are fragmented. I like longer paragraphs, each 130-160 words,
  with strong topic sentences. A reader could sketch the topic sentences only
  to get a quick idea about the paper.}

\jy{Three questions to address in introduction and in this order:
  why important; what've been done; what's new.}

Mixed doubles curling introduces a fundamentally different strategic
environment than traditional curling, making data-driven analysis particularly
important. It was developed at the International Olympic Committee's request
for a faster, higher-variance format that is less likely to settle into fixed
patterns. The modified free guard zone (MFGZ), two pre-placed stones, hammer
transfer on blanks, eight-end matches, and the power play all alter tactical
incentives and end-level planning. These rule changes keep more stones in play
and raise the stakes of single ends. Mixed doubles' novelty, however, means
that little-to-no data-driven strategy exists yet for the sport.

The power play is the most consequential strategic innovation. Each team may
use it once, but only when it has the hammer; the pre-placed stones are moved
to the sides, opening the center of the sheet. This shift changes incentives
for both teams and demands clear guidelines on when to call the power play, how
to execute it, and how to respond.
% briefly outline what the purpose of the data challenge is exactly -> there isn
% o data anlysis done on mixed doubles curling all mixed doubles curling
% strategy and intuition relies on strategy formulated from the tradtional game.
% this is why we have the data challenge, is so that we can develop strategy
% independent of the 4 person game.

% the key focus of the data challenge is the power play, for which
% we have no reliable strategy because this is a novel rule
% powerplay questions: how and when and why is it used
% how should you use it or or respond to it etc
Previous research has been conducted on the topic of curling strategy,
primarily with the goal of confirming or disproving well-known curling ideas.
\citet{willoughby2005analysis} analyzed an important strategic decision in
curling—whether to blank an end or score a single point in later ends. He
observed differences in shot selection between North American and European
teams and concluded that blanking is generally the superior option.
\citet{clement2012analysis} extended this analysis by incorporating the free
guard zone and women’s match data, finding that blanking may not be optimal in
the ninth end depending on the score differential. Also, the statistical
significance of hammer possession and end outcome has been
observed\citep{park2013curling}. Overall, previous research addresses
curling-specific problems that do not apply to mixed doubles curling. The
various rule changes and addition of the power play prompt our analysis on the
unique aspects of mixed doubles curling.

Using data from three World Championships and one Winter Olympics, this paper
offers the first analysis of strategic behavior surrounding the power play in
professional mixed doubles curling. We show that late-phase behavior shifts
toward defense in power play ends, with the largest and most persistent shift
coming from the non-calling team. We quantify timing effects, provide
team-level descriptive strategy profiles, and validate the behavioral proxy
with stone-removal outcomes. We then focus on non-hammer team behavior by
analyzing common counter power play shot types and show that different openings
do not significantly alter end-level scoring outcomes, even though they produce
meaningful differences in execution quality and cumulative score dynamics.
Despite teams systematically varying shot choice with game context, the data
provide no evidence that any single counter-strategy is objectively dominant
when opening an end against the power play.
% the key focus of the data challenge is the power play, for which we have no
% reliable strategy because this is a novel rule powerplay questions: how and
% when and why is it used how should you use it or For respond to it etc

% THESES:
% what behavior is seen during powerplay -> these behaviors are seen -> do they
% lead to anything

% Defensive behavior is observed during powerplay typically in later ends,
% strategic trends are not apparent across teams or championships. 

% The different types of shots to counter the powerplay do not have significant
% differences in expected HT or NHT score but they have significant differences
% in execution score & cumulative score differences. -> Teams tend to use
% different shots in different cumulative score differences, but the data
% suggests there is no reason to prefer different behavior when opening an end
% as NHT during Power play.

\section{Data}

We use Curlit's mixed doubles tracking data, which records shot-by-shot stone
positions and analyst annotations across three World Mixed Doubles
Championships and one Winter Olympics. The dataset contains 26,730 shot rows
and includes unique identifiers for competition, game, end, team, player, and
shot order, so that each match can be reconstructed chronologically. In
addition to x--y stone coordinates after every throw, Curlit provides
analyst-graded fields for intended task and execution quality, which we use to
classify behavior.

This is the first publicly available mixed doubles dataset that combines
stone-level geometry with end-level context and shot assessment. These features
allow us to link strategic choices to both sheet state and game state. End
results are recorded for both teams, which enables reconstruction of score
progression and pre-end score differential.

At the shot level, the distribution is draw-heavy, consistent with the MFGZ and
an opportunistic or probing early phase. Figure~\ref{fig:thrown_positions_all}
summarizes the spatial density of thrown stone locations across all shots.
Power play usage is observed in a minority of ends and varies by end number,
and task-group shares shift between power play and non-power-play ends
(Figure~\ref{fig:data_pp_summary}).

\begin{figure}[htbp]
  \centering
  \includegraphics[width=0.6\textwidth]{img/thrown_positions_all.png}
  \caption{Stone position density across all shots, illustrating the regions
    of the sheet most frequently occupied after a throw.}
  \label{fig:thrown_positions_all}
\end{figure}

\begin{figure}[htbp]
  \centering
  \begin{subfigure}[b]{0.47\textwidth}
    \centering
    \includegraphics[width=\linewidth, height=0.23\textheight,
      keepaspectratio]{img/powerplay_end_frequency.png}
    \caption{Power play usage by end number.}
    \label{fig:powerplay_end_frequency}
  \end{subfigure}
  \hfill
  \begin{subfigure}[b]{0.47\textwidth}
    \centering
    \includegraphics[width=\linewidth, height=0.23\textheight,
      keepaspectratio]{img/task_by_powerplay.png}
    \caption{Task-group shares by power play status.}
    \label{fig:task_by_powerplay}
  \end{subfigure}
  \caption{Power play usage context and task composition.}
  \label{fig:data_pp_summary}
\end{figure}

% Data collection was that Curlit analysts entered the stone positions into
% the software and evaluated shot quality on objective criteria.

% Data collection was that curlit analysts entered the stone positions onto a
% softeware and evaluated shot quality on objective criteria (awarding points)

% 3 world champsionships 1 winter olympics (beijing 2022)

% This is the first data set that:
% 1. Contains stone-level coordiante data for geometric analyses
% 2. Contains powerplay-end level data for mixed doubles curling
% 3. Objective qualifiers for shot-type and shot-execution

% basic histograms which describe the data, e.g. endid, powerplay 0,1,2, count
% normal/powerplay ends how many shots every noc hits how often each shot is
% shots

\section{Power Play and Late-Phase Shot Selection}
\subsection{Data}
\subsection{Methodology}

Conceptually, we define behavior as offensive, defensive, or probing
(opportunistic). These behaviors can appear tactically within a single end
(shot-to-shot choices) and strategically across ends (an intended end goal or
game plan). Because probing behavior is inherently reactive and not uniquely
encoded in the task labels, we treat neutral or ambiguous shots as the closest
empirical proxy for probing in this analysis.

Our behavioral analysis is conducted at the team-end level, which creates a
paired structure within each end and allows a direct comparison between the
power play caller and the opponent. We focus on the late phase of the end (the
final seven deliveries) because early shots are heavily constrained by the
Modified Free Guard Zone and the pre-placed stones. Late shots therefore
provide the clearest window into strategic choice rather than rule-driven
necessity.

Behavior is operationalized through shot selection, using the task labels
assigned by analysts. We group tasks into offensive placements (draws, guards,
raises, freezes, and hit-and-rolls), defensive removals (peels, take-outs,
clears, doubles, and promotions), and a neutral category for throughs or
no-shot entries. This choice is deliberate: the task coding is the only
consistently available, interpretable indicator of intent at the shot level.
Position-based intent would require model-driven inference and introduce a
second layer of assumptions, so we treat shot selection as the observable
behavioral signal and reserve positional data for validation.

Power play exposure is assigned at both the end and team-end levels. An end is
marked as a power play end if either team uses its power play, and within such
ends we identify the caller and the non-caller. This yields a three-category
comparison group: non-power-play ends, power-play callers, and power-play
non-callers. Each team-end is also annotated with hammer status, pre-end score
differential, end number, team identity, and match identity to control for
state and repeated context.

For each team-end we compute late-phase shares of offensive, defensive, and
neutral shots and define the behavior index as defensive share minus offensive
share, where larger values indicate a more defensive profile. We also split the
late phase into early-late (shots 16--18) and late-late (shots 19--22) to test
timing mechanisms. Because hammer teams throw four late shots and non- hammer
teams throw three, shares are in discrete increments; using shares standardizes
exposure across teams and ends.

The primary specification is fixed-effects OLS with clustering at the match
level. Fixed effects for team and match absorb persistent strategy and match-
specific conditions, while end fixed effects control for systematic end-of-
game dynamics. This specification favors interpretability of the power play
coefficients in share points and is conservative with respect to within-match
correlation. Diagnostics include residual plots, Q--Q plots, and influence
measures. As a validation, we test whether higher defensive shares correspond
to more opponent stones removed during late shots, using the off-sheet sentinel
to identify removals. Limitations include the discrete nature of the outcome,
potential selection into power play usage, and reliance on task coding as a
proxy for intent.

Formally, the main model is:
\[
  \begin{aligned}
    \text{behavior\_index}_{i e} =\; & \beta_0 + \beta_1 \text{PP\_caller}_{i e}
    + \beta_2 \text{PP\_noncaller}_{i e}                                                     \\
                                     & + \beta_3 \text{has\_hammer}_{i e}
    + \beta_4 \text{pre\_end\_score\_diff}_{i e}                                             \\
                                     & + \alpha_e + \gamma_i + \delta_m + \varepsilon_{i e},
  \end{aligned}
\]
where $i$ indexes team-ends, $e$ indexes ends, and $m$ indexes matches. The
omitted reference category is non-power-play ends, and standard errors are
clustered at the match level. We estimate the same specification for the lateA
and lateB timing splits.

% [Power Play intent results; source: notebooks/behavior_classes.qmd]
\subsection{Results}

Across the late-phase sample (5,274 team-ends), non-power-play ends are
predominantly offensive. The mean defensive share is 0.295 while the offensive
share is 0.690, yielding a mean behavior index of $-0.395$. Power-play ends
shift toward defense for both teams. For callers, the defensive share rises to
0.426 and the offensive share falls to 0.566 (behavior index $-0.141$). For
non-callers, the defensive share is 0.455 and the offensive share is 0.540
(behavior index $-0.085$). These descriptive contrasts show a clear defensive
shift in power-play ends, with the larger shift occurring for the opponent.

Power play usage is concentrated in non-tied score states: it is used in 0.113
of tied ends versus 0.265 when trailing or leading. Within power-play ends, the
defensive shift is strongest in protective contexts. For example, when the
caller is leading, the caller's defensive share rises to 0.658 and the
non-caller's rises to 0.587. Timing splits show that the caller's adjustment
appears early, while the non-caller remains defensive throughout: in lateA,
defensive shares are 0.266 (non-PP), 0.417 (caller), and 0.457 (non-caller),
and in lateB they are 0.314, 0.434, and 0.455, respectively.
Figure~\ref{fig:pp_def_share_by_score} summarizes the score-state pattern by
plotting mean late-phase defensive share across trailing, tied, and leading
contexts for each power-play group.

\begin{figure}[htbp]
  \centering
  \includegraphics[width=0.8\textwidth]{img/pp_def_share_by_score.pdf}
  \caption{Mean late-phase defensive share by score state, with separate lines
    for non-power-play ends, power-play callers, and power-play non-callers.}
  \label{fig:pp_def_share_by_score}
\end{figure}

The fixed-effects regression corroborates these descriptive patterns after
controlling for hammer, score differential, end, team, and match
(Table~\ref{tab:pp_regression}). In the main model, the PP caller coefficient
is 0.068 ($p<0.05$) while the PP non-caller coefficient is 0.449 ($p<0.01$),
indicating a much larger opponent response. Timing models show a caller effect
in lateA (0.128, $p<0.01$) but not in lateB (0.009, $p=0.82$), while the
non-caller remains strongly defensive in both lateA (0.516) and lateB (0.414).

Positional validation strengthens the behavioral interpretation: the Spearman
correlation between defensive share and opponent stones removed late is 0.804.
Taken together, the results indicate that power-play ends induce a defensive
shift for both teams, with a substantially larger and more persistent response
from the non-caller.

\begin{table}[t]
  \centering
  \caption{Power play effects on late-phase behavior index}
  \label{tab:pp_regression}
  \begin{tabular}{lccc}
    \hline
    & Main & Mid & Close \\
    \hline
    PP caller & 0.068** & 0.128*** & 0.009 \\
    & (0.030) & (0.038) & (0.038) \\
    PP non-caller & 0.449*** & 0.516*** & 0.414*** \\
    & (0.035) & (0.046) & (0.043) \\
    \hline
    N & 5,274 & 5,274 & 5,274 \\
    R$^2$ & 0.232 & 0.221 & 0.161 \\
    End FE & Yes & Yes & Yes \\
    Team FE & Yes & Yes & Yes \\
    Match FE & Yes & Yes & Yes \\
    \hline
  \end{tabular}
  \begin{flushleft}
  \footnotesize Notes: Cluster-robust standard errors at the match level in
  parentheses. $^{*}$ $p<0.10$, $^{**}$ $p<0.05$, $^{***}$ $p<0.01$.
  \end{flushleft}
\end{table}


To address team strategy profiles, Table~\ref{tab:team_profiles} reports
descriptive late-phase behavior by team. These profiles summarize each team’s
late-phase defensive share, offensive share, and behavior index, alongside
power-play calling rate and hammer frequency. They are intended as descriptive
benchmarks rather than causal estimates.

\begin{table}[t]
  \centering
  \caption{Team strategy profiles (late-phase behavior and power play usage)}
  \label{tab:team_profiles}
  \begin{tabular}{lrrrrrrr}
    \hline
    NOC & TeamID & N & Def. & Off. & Index & PP call & Hammer \\
    \hline
    JPN & 16 & 226 & 0.367 & 0.621 & -0.253 & 0.124 & 0.504 \\
    HUN & 36 & 71 & 0.367 & 0.622 & -0.255 & 0.127 & 0.535 \\
    EST & 37 & 263 & 0.360 & 0.624 & -0.264 & 0.122 & 0.510 \\
    ITA & 24 & 311 & 0.356 & 0.629 & -0.274 & 0.090 & 0.434 \\
    FRA & 13 & 80 & 0.343 & 0.626 & -0.283 & 0.125 & 0.600 \\
    USA & 20 & 311 & 0.350 & 0.641 & -0.290 & 0.103 & 0.437 \\
    SUI & 18 & 302 & 0.343 & 0.637 & -0.293 & 0.099 & 0.493 \\
    SWE & 19 & 306 & 0.349 & 0.643 & -0.294 & 0.105 & 0.493 \\
    CAN & 10 & 313 & 0.342 & 0.644 & -0.301 & 0.089 & 0.489 \\
    SCO & 14 & 247 & 0.340 & 0.646 & -0.306 & 0.109 & 0.506 \\
    ESP & 34 & 203 & 0.340 & 0.651 & -0.311 & 0.123 & 0.542 \\
    TUR & 51 & 218 & 0.333 & 0.649 & -0.315 & 0.124 & 0.528 \\
    AUS & 46 & 300 & 0.338 & 0.654 & -0.317 & 0.113 & 0.487 \\
    CZE & 22 & 296 & 0.332 & 0.649 & -0.317 & 0.118 & 0.493 \\
    CHN & 43 & 220 & 0.324 & 0.659 & -0.335 & 0.123 & 0.509 \\
    KOR & 44 & 218 & 0.321 & 0.666 & -0.345 & 0.119 & 0.491 \\
    DEN & 11 & 219 & 0.296 & 0.689 & -0.393 & 0.123 & 0.530 \\
    NZL & 30 & 132 & 0.294 & 0.691 & -0.397 & 0.129 & 0.553 \\
    NOR & 17 & 339 & 0.293 & 0.694 & -0.401 & 0.115 & 0.469 \\
    GER & 15 & 208 & 0.294 & 0.695 & -0.401 & 0.120 & 0.543 \\
    ENG & 23 & 60 & 0.297 & 0.703 & -0.406 & 0.133 & 0.583 \\
    NED & 25 & 207 & 0.272 & 0.711 & -0.439 & 0.121 & 0.536 \\
    FIN & 12 & 66 & 0.254 & 0.739 & -0.485 & 0.136 & 0.485 \\
    GBR & 27 & 86 & 0.243 & 0.739 & -0.496 & 0.116 & 0.512 \\
    AUT & 28 & 72 & 0.233 & 0.756 & -0.523 & 0.111 & 0.486 \\
    \hline
  \end{tabular}
  \begin{flushleft}
  \footnotesize Notes: Def. and Off. are late-phase defensive and offensive shot shares. Index = Def. - Off. PP call is the share of team-ends in which the team called power play. Hammer is the share of team-ends with hammer.
  \end{flushleft}
\end{table}


% 1st - summarize 2nd - why is your approach the best 3rd - limitations. what
% question might someone else want to answer that we can't address

% bring up the most interesting and insightful results bring up the
% implications, make actionable recommendations

% alejandro: curling strategy says to enter with a game-plan game-plans for how
% to use powerplay are not observed with this data. 

% [Power Play intent findings; updated discussion of methods + results]
\subsection{Discussion}

Analysts describe the power play as a mechanism to clear the center of the
sheet by moving the pre-placed stones to the side, which should make offense
easier. At the same time, practitioner intuition (e.g., CurlTech and Rocks
Across the Pond) suggests that teams often invoke the power play late and in a
defensive posture. Our results reconcile these views. The late-phase behavior
shift is decisively toward defense for both teams, and the non-caller’s shift
is substantially larger and more persistent than the caller’s. This confirms
the analysts’ intuition that power play does not simply amplify offense and, in
practice, often corresponds to conservative late-phase shot selection. It also
disconfirms the common expectation that the hammer team becomes more offensive
in power play ends; the hammer-only shift is modest compared to the opponent’s
response.

The state dependence of usage supports a strategic interpretation. Power play
is used much less when tied and more when teams are either trailing or leading,
precisely the contexts in which variance management matters most. The cleared
sheet lowers the cost of removal play and raises the risk of large swings, so
both teams appear to prioritize control once takeouts are available. The
caller’s early defensive adjustment is consistent with stabilizing the end,
while the non-caller’s sustained defensive profile suggests a deliberate
attempt to neutralize the caller’s positional leverage in open ice.

These patterns translate into actionable guidance. For the hammer team, power
play should be treated as a situational control tool rather than a guaranteed
offensive boost. If the objective is to protect a lead, early defensive
sequencing appears effective; if trailing, teams should anticipate a strong
defensive counter and build scoring chances early before removal play
dominates. For the non-hammer team, the evidence supports proactive disruption
and removal as the default response, with a clear plan for when to transition
from probing offense to conservative control.

Methodologically, we interpret behavior through shot selection because it is
the only directly observed indicator of intent; positional validation supports
this proxy, but the estimates are behavioral associations rather than causal
effects. This framing sets up the clustering-based discussion that follows by
providing an end-level interpretation of power play behavior before we examine
the non-hammer team’s first-shot response patterns.

\section{First-shot Non-hammer Team Powerplay Adaptation}
\subsection{Data}
\subsection{Methodology}
% Mark std. coordinates use HDBScan for clustering -> visualize w/ PCA, tSNE
% cluster quality evaluated on coverage, score, and cluster stability use welch
% ANOVA and bootstrap, use post-hoc test to determine differences between
% clusters hypothesis test: test if there is at least one group that is
% significantly different from another

% assumptions: ind, approx. normal (for anova)
The following is a brief summary of the preprocessing steps applied prior to
analysis. After the coordinates were standardized, the stone-level data were
merged with the end-level data and filtered to the first shot of the power play
right-formation end. The columns for stones 3 to 6 and 9 to 12 were dropped
because they can be played on the first shot. Columns for stone 2 and 8 were
combined and shifted into one column for stone 2 since both belong to the
non-hammer team. As for stones 1 and 7, both columns represented pre-placed
stones, although the designation of guard or house stone was unclear for each
row. The coordinates were swapped when either stone 1 had coordinates for the
hammer pre-placed house stone or stone 7 had coordinates for the non-hammer
guard. Finally, rows with missing values were dropped. This excludes shots that
removed any stone before the Modified Free Guard Zone.

Next, to identify patterns in stone positions, PCA, t-SNE and UMAP were applied
to the 6-dimensional coordinate data consisting of the x and y coordinates for
stones 1, 2, and 7. HDBSCAN was selected because non-circular clusters appear
in the 2-D visualizations from graphing two components of PCA, t-SNE, and UMAP.
To optimize clustering performance, parameters of minimum cluster size = 5 and
min samples = 5 were chosen from a grid search. Cluster quality was then
assessed using DBCV score, coverage, silhouette score with noise excluded, and
cluster membership probabilities. Finally, the clusters were visualized by
color on 2-D components of the dimensionality reduction techniques to confirm
cluster separation.

Histograms and bar charts were used to visualize the distributions and counts
of key variables. Curling sheet scatter plots were also used to visualize the
positions of each stone by cluster and task. Following visualization,
performance metrics across clusters were compared using Welch’s one-way ANOVA
to account for unequal group sizes and variances. Some variables were highly
skewed, which may slightly affect the p-values, but bootstrap confidence
intervals were used for the means to provide robust estimates. Where the
overall ANOVA indicated significant differences, post hoc pairwise comparisons
were performed using the Games–Howell test. Cluster groups were defined as C0
(n = 27), C1 (n = 73), and C2 (n = 62). Values are presented as group means
with 95\% bootstrap confidence intervals to account for the skewed
distributions. Detailed results for all variables are provided in
Table~\ref{tab:cluster_test}.

To observe different countries' first non-hammer shot selection in the power
play, bar charts are created on both left and right power-play formation ends.
Teams are only included if they have played more than twenty power play ends.
Based on the results of the clustering, similar type of shots are grouped
together. The top five countries for each group of shot type are plotted to
compare play styles in response to the power play.

\subsection{Results}
The HDBSCAN clustering (minimum cluster size = 5, minimum samples = 5) produced
four clusters, with a coverage rate of 0.761 and a DBCV score of 0.428. These
metrics suggest reasonable clustering structure with a fair amount of noise.
Cluster sizes ranged from 7 to 76 with 55 points labeled as noise. Cluster
assignments from HDBSCAN were probabilistic. Clusters 0–3 had mean membership
probabilities of 0.773, 0.753, 0.891, and 0.982, respectively, indicating that
Cluster 3 contained the most tightly grouped points, while Clusters 0 and 1 had
weaker assignments. Fifty-five points were labeled as noise, reflecting points
not clearly assigned to any cluster. Clusters visualized on 2-D components of
PCA, t-SNE and UMAP revealed well separated clusters with minimal overlap
(Figure~\ref{fig:fs_tSNE}). With noise excluded, the silhouette score was
calculated as 0.604, suggesting reasonable to strong clustering structure.

\begin{figure}[t]
  \centering
  \includegraphics[width=1.0\textwidth]{img/fs_t-SNE.pdf}
  \caption{“t-SNE visualization of the first-shot power-play left-formation
    dataset in two dimensions.
    Each point represents a sample, colored by cluster.
    Clusters indicate similarity in the feature space.”}
  \label{fig:fs_tSNE}
\end{figure}

To further characterize these clusters, we examined the spatial distribution of
stone placements and shot types for each cluster. Plotting the stone
coordinates on a curling sheet grouped by cluster reveals three clear board
positions, shown in Figure~\ref{fig:C_all}. As illustrated in
Figure~\ref{fig:C0}, the Cluster~0 pattern corresponds to situations in which
the first non-hammer team shot collides with the non-hammer guard, pushing the
stone into or near the house. Figure~\ref{fig:C1} shows that Cluster~1 is
characterized by the pre-placed stones remaining untouched, with the first shot
typically landing around the hammer team’s house stone. Similarly, Cluster~2
(Figure~\ref{fig:C2}) exhibits untouched pre-placed stones, but with the first
shot positioned in the front of the house or even higher. Cluster~3 was
identified as a distinct central draw pattern but contained a very small sample
size ($n=7$); despite strong cluster membership probabilities, it was excluded
from further analysis due to its limited representation.

\begin{figure}[t]
  \centering

  \begin{subfigure}[b]{0.32\textwidth}
    \centering
    \includegraphics[width=\textwidth]{img/C0.pdf}
    \caption{Cluster 0: Tick shot}
    \label{fig:C0}
  \end{subfigure}
  \hfill
  \begin{subfigure}[b]{0.32\textwidth}
    \centering
    \includegraphics[width=\textwidth]{img/C1.pdf}
    \caption{Cluster 1: Draw}
    \label{fig:C1}
  \end{subfigure}
  \hfill
  \begin{subfigure}[b]{0.32\textwidth}
    \centering
    \includegraphics[width=\textwidth]{img/C2.pdf}
    \caption{Cluster 2: Front/Guard}
    \label{fig:C2}
  \end{subfigure}

  \caption{Coordinate visualizations after the first shot in the power-play
    left formation. Each panel highlights a different cluster, which can be
    identified as shot types: tick (Panel A), draw around the guard (Panel B),
    and front/guard (Panel C).}
  \label{fig:C_all}
\end{figure}

We next examined the distribution of shot types within each cluster. Histograms
of proportion of each task grouped by cluster conclude that Cluster 0 having a
roughly equal proportion of raise/tapbacks and wick/soft-peeling shots. Cluster
1 only consists of draws, and Cluster 2 has an equal number of fronts and
guards.

Finally, we assessed whether these cluster groupings corresponded to
differences in game outcomes. Cumulative score difference differed
significantly across clusters (Welch’s ANOVA, $p=0.013$;
Table~\ref{tab:cluster_test}), with the highest mean observed in C1 (6.16, 95\%
CI 5.26--7.08) and the lowest in C0 (3.44, 95\% CI 1.89--5.04). Post-hoc
Games–Howell tests indicated a significant difference between C0 and C1
($p=0.016$), while differences between C0 and C2 ($p=0.343$) and C1 and C2
($p=0.205$) were not statistically significant. Points also varied across
clusters (Welch’s ANOVA, $p=0.005$; Table~\ref{tab:cluster_test}), with a
significant post hoc difference between C1 and C2 ($p=0.004$), but not between
C0 and C1 ($p=0.326$) or C0 and C2 ($p=0.845$). Hammer and non-hammer scores
did not differ significantly across clusters (Welch’s ANOVA, $p=0.769$ and
$p=0.833$, respectively; Table~\ref{tab:cluster_test}). Overall, these results
indicate that cumulative score difference and points showed cluster-specific
variation, while hammer-related outcomes were similar across groups.

\jy{n is not $n$}
\begin{table}[t]
  \centering
  \includegraphics[width=0.8\textwidth]{img/cluster_test.pdf}
  \caption{Group means (with 95\% bootstrap confidence intervals) for each
    outcome variable by HDBSCAN cluster. p-values are from Welch’s one-way
    ANOVA. Significant pairwise differences were assessed using Games–Howell
    post hoc tests.}
  \label{tab:cluster_test}
\end{table}

\subsection{Discussion}
Analysts have observed different responses for countering the power play,
although there have been no studies to identify exactly which shots teams hit
and their effect on performance and context metrics. By clustering the board
positions after the non-hammer team hits its first shot, we can observe three
distinct patterns in curling mixed doubles championships. The tick, guard, and
draw around the guard are all common shots that occur in response to the power
play.

Welch ANOVA was used to compare differences between clusters, revealing
significant differences in cumulative score difference and points in non-missed
shots, and no significant differences in hammer or non-hammer score. Post hoc
Games–Howell tests show significant cumulative score differences between the
tick and draw around the guard. All shots are utilized when the hammer team has
the score advantage. However, the tick shot is thrown when the score difference
is smaller, while the draw is thrown when the hammer team has a blowout lead.

Teams may have tendencies to throw less risky shots when they are behind less
in the score compared to when they are down many points. Nevertheless, as there
are no significant differences in hammer and non-hammer score between the
clusters, shot choice may not have an important role in end score performance.
We suggest that when countering the power play as the non-hammer team, teams
choose the starting player based on their overall skill in executing one of the
three options observed.

When including all of the power play ends, certain countries play the draw,
guard or tick shot a lot more than others. The bar chart of proportions of task
groups by nation indicates the clear difference in strategy in throwing the
first stone during power play ends (Figure~\ref{fig:noc_pptask}). The tasks are
grouped by the results discovered through clustering. All teams tend to hit
more draws and tick shots, with Korea hitting 67\% of their shots as draws.
Germany plays the most guards at 48\%. Sweden along with Scotland hit a large
majority of their shots as Raise/Tapbacks and Wick/Soft-Peels. These
differences confirm the analysts' intuition about varying responses to
countering the power play.

\begin{figure}[t]
  \centering
  \includegraphics[width=\textwidth]{img/noc_pptask.pdf}
  \caption{“Proportions of grouped tasks hit by country. Data includes all
    power play ends and is limited to the first stone. Only the top five teams
    with more than 20 power play ends are shown.”}
  \label{fig:noc_pptask}
\end{figure}

% \subsection{Limitations}
\jy{Could be brief, because this is not exactly an academic paper but a
  report for reference for the coaching team.}

% lack of powerplay sample size very skewed data. task, points, expected hammer
% points data quality. i.e. inconsistent coordinate assignments, incosistent
% column assignments

% freeze vs draw inconsistency 

% imbalanced data with logical end devisions before vs after MFGZ 

% one of the clusters does not identify raise / tap-backs on the hammer stone

% the draw cluster has a very small sample size but is recognized as a cluster

% Limitations include small power play sample size and heavily skewed data in
% multiple columns such as task labels, execution points, and hammer score.
% Coordinates and columns were sometimes assigned inconsistently; for example, a
% draw may be identified as a freeze.

\section{Conclusion}
% quick summary
We find that the power play is associated with a consistent late-phase shift
toward defensive shot selection for both teams, with the non-caller showing the
larger and more persistent adjustment. These shifts are concentrated in
non-tied score states and align with a control-oriented interpretation of the
open sheet. The behavioral proxy based on task selection is supported by
positional validation, suggesting it captures meaningful intent under the data
constraints.

At the shot level, clustering of the non-hammer team’s first response reveals
three dominant counterplay patterns. These include the tick shot, draw around
the guard and a center guard. We conclude that while shot execution and score
difference when executed varies, the first stone shot selection does not impact
non-hammer or hammer-team end scores. While teams do show a clear strategy for
their first stone, it is likely other factors drive powerplay end-level
success.
%\section{Appendices} if there is a need for code appendix or otherwise, we'll
%put it at the botton.

\jy{Do you have team strategy profiles?}
\bibliographystyle{apalike}
\clearpage
\bibliography{references.bib}
%i have my latex in vscode compiled to use natbib, not bibtex. it will work for
% you but i havent got mine working yet, once i set it back we will use APA

\end{document}
