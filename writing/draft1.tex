\documentclass{article}
\usepackage{amsmath}
\usepackage[margin=1in]{geometry}
\usepackage{graphicx}
\usepackage[utf8]{inputenc}
\usepackage{natbib}
\usepackage{xcolor}

\title{Analysis of Power Play Behavior and Counter Play in Mixed Doubles Curling\\
  \large CSAS 2026 Data Challenge Submission}

\date{[15 January 2026]}



\begin{document}

\begin{titlepage}
  \maketitle
  % i may add a table of contents here.
\end{titlepage}

% Note: % is comment character, \ backslash is the escape character (needed for $, %, etc.)

% New paragraphs will be formed automatically similar to quarto, just leave a blank line in between
% lines. Inline math can be called by wrapping with $, for example $ math equation$.


\section{Introduction}

% briefly outline what the purpose of the data challenge is exactly
% -> there isn o data anlysis done on mixed doubles curling
% all mixed doubles curling strategy and intuition relies on
% strategy formulated from the tradtional game.
% this is why we have the data challenge, is so that we can develop
% strategy independent of the 4 person game.

% the key focus of the data challenge is the power play, for which
% we have no reliable strategy because this is a novel rule
% powerplay questions: how and when and why is it used
% how should you use it or or respond to it etc

% THESES:
% what behavior is seen during powerplay -> these behaviors are seen -> do they lead to anything

% Defensive behavior is observed during powerplay typically in later ends, strategic trends
% are not apparent across teams or championships. 

% The different types of shots to counter the powerplay do not have significant differences
% in expected HT or NHT score but they have significant differences in execution score
% & cumulative score differences. 
% -> Teams tend to use different shots in different cumulative score differences, but
% the data suggests there is no reason to prefer different behavior when opening an end as NHT
% during Power play.


\subsection{Literature Review}

% bring up scarcity of resources and limitations of extant literature

\section{Data}

% source is curlit

% Data collection was that curlit analysts entered the stone positions onto a softeware and evaluated
% shot quality on objective criteria (awarding points)

% 2 world champsionships 1 winter olympics (beijing 2022)

% This is the first data set that:
% 1. Contains stone-level coordiante data for geometric analyses
% 2. Contains powerplay-end level data for mixed doubles curling
% 3. Objective qualifiers for shot-type and shot-execution

% basic histograms which describe the data,
% e.g. endid, powerplay 0,1,2, count normal/powerplay ends
% how many shots every noc hits
% how often each shot is shots


\section{Methodology}
\subsection{Power Play Behavior Clustering}
% Alejandro
% std. coordinates
% estimate behavior class binary on end-level predictor set

% finding of mine was that behavior 


% Bridge section flowwwww

% The behavior of the HT is much easier to identify during power play
% 

% Once power play behavior has been identified, clustering analysis
% for first-stone level data is used to identify ideal response
% types by the NHT


\subsection{First-shot NHT Clustering}

% Mark
% std. coordinates
% use HDBScan for clustering -> visualize w/ PCA, tSNE
% cluster quality evaluated on coverage, score, and cluster stability
% use welch ANOVA and KW-test, use post-hoc test to determine differences between clusters
% hypothesis test: test if there is at least one group that is significantly
% different from another

% assumptions: ind, approx. normal (for anova), skewed (for kw-test)

\section{Results}

% objective reports. figures go here. no need for fluff

\section{Discussion}

% 1st - summarize
% 2nd - why is your approach the best
% 3rd - limitations. what question might someone else want to answer that we can't address


% bring up the most interesting and insightful results
% bring up the implications, make actionable recommendations

% alejandro: curling strategy says to enter with a game-plan
% game-plans for how to use powerplay are not observed with this
% data. 

\subsection{Limitations}
% lack of powerplay sample size
% very skewed data. task, points, expected hammer points
% data quality. i.e. inconsistent coordinate assignments, incosistent column assignments

% freeze vs draw inconsistency 

% imbalanced data with logical end devisions before vs after MFGZ 

% one of the clusters does not identify raise / tap-backs on the hammer stone

% the draw cluster has a very small sample size but is recognized as a cluster


\subsection{Conclusion}
% quick summary

%\section{Appendices}
%if there is a need for code appendix or otherwise, we'll put it at the botton. 
\bibliographystyle{plainnat}
\bibliography{../references}
%i have my latex in vscode compiled to use natbib, not bibtex. it will work for you
% but i havent got mine working yet, once i set it back we will use APA

\end{document}

