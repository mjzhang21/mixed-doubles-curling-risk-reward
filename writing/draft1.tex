% LTeX: language=en-US
\documentclass[titlepage, 12pt]{article}
\usepackage{amsmath}
\usepackage[margin=1in]{geometry}
\usepackage{graphicx}
\usepackage[utf8]{inputenc}
\usepackage{natbib}
\usepackage{xcolor}
\usepackage{caption}
\usepackage{subcaption}  

\newcommand{\jy}[1]{{\textcolor{violet}{JY:\@ #1}}}


\title{Strategic Adaptation to the Power Play in Mixed Doubles Curling}

% \date{[15 January 2026]}



\begin{document}
\maketitle

\begin{abstract}
  The power play in mixed doubles curling was introduced to open
  the sheet and encourage offensive scoring, yet its strategic
  effects remain largely undocumented. Using stone-level
  coordinate and shot assessment data from three World Mixed
  Doubles Curling Championships and one Winter Olympics, this
  study examines strategic behavior associated with power play
  usage and response. End-level intent is inferred from
  shot-level task information by distinguishing early and late
  phases defined by the Modified Free Guard Zone, combining
  rule-based classification with unsupervised clustering to
  characterize offensive and defensive patterns. Across more
  than 5,000 team-ends, power play ends show a systematic shift
  toward defensive behavior in the late phase relative to
  non-power-play ends, particularly in score-protective
  contexts, and this shift is not driven solely by hammer-team
  behavior. Analysis of the non-hammer team’s first shot against
  the power play identifies three dominant counterplay patterns
  based on spatial stone configurations. These patterns differ
  in execution quality and cumulative score dynamics but do not
  yield statistically significant differences in end-level
  scoring outcomes.
\end{abstract}

% Note: % is comment character, \ backslash is the escape character (needed for $, %, etc.)

% New paragraphs will be formed automatically similar to quarto, just leave a
% blank line in between lines. Inline math can be called by wrapping with $, for
% example $ math equation$.

\section{Introduction}

% briefly outline what the purpose of the data challenge is exactly
% -> there isn o data anlysis done on mixed doubles curling
% all mixed doubles curling strategy and intuition relies on
% strategy formulated from the tradtional game.
% this is why we have the data challenge, is so that we can develop
% strategy independent of the 4 person game.

\jy{Paragraphs are fragmented. I like longer paragraphs, each 130-160 words,
  with strong topic sentences. A reader could sketch the topic sentences only
  to get a quick idea about the paper.}

\jy{Three questions to address in introduction and in this order:
  why important; what've been done; what's new.}

Mixed Doubles Curling introduces a fundamentally different strategic
environment than traditional curling, making data-driven analysis particularly
important. Mixed Doubles Curling is the product of the International Olympic
Committee's request for a variation of the game that is more fast-paced,
dynamic, high-risk, and less likely to fall into specific patterns determined
by a game strategy. The unique rules of the Mixed Doubles game including the
Modified Free Guard Zone (MFGZ), two pre-placed stones, the hammer being
transferred even on a blank, only eight ends, and of course the power play,
necessitate new tactical and strategic decision-making on the ice. These
modifications are intended to keep more stones in play, increasing the odds of
big ends and raising the stakes for a single end. Mixed Doubles' novelty,
however, means that little-to-no data-driven strategy exists yet for the sport.

The introduction of the power play represents the most consequential strategic
innovation in Mixed Doubles Curling. Of particular note is the power play, a
concept entirely new to curling. The power play can be used by each team once
per team if they have the hammer. In power play, the pre-placed stones, which
usually are a front guard and a button scoring stone for the non-hammer and
hammer team respectively, are moved off to the side. This opens up the sheet
and requires a fundamentally different approach. But if there is anything to
learn from traditional curling, a strategic approach is always more effective
than playing pure tactics; and thus at the professional level it is the teams'
best interest that proper strategy on when to use the power play, how to use
it, and how to respond to it, is developed.
% briefly outline what the purpose of the data challenge is exactly -> there isn
% o data anlysis done on mixed doubles curling all mixed doubles curling
% strategy and intuition relies on strategy formulated from the tradtional game.
% this is why we have the data challenge, is so that we can develop strategy
% independent of the 4 person game.

% the key focus of the data challenge is the power play, for which
% we have no reliable strategy because this is a novel rule
% powerplay questions: how and when and why is it used
% how should you use it or or respond to it etc
Previous research has been conducted on the topic of curling strategy,
primarily with the goal of confirming or disproving well-known curling ideas.
Willoughby \citet{willoughby2005analysis} analyzed an important strategic
decision in curling—whether to blank an end or score a single point in later
ends. He observed differences in shot selection between North American and
European teams and concluded that blanking is generally the superior option.
Clement \citet{Clement2012analysis} extended this analysis by incorporating the
free guard zone and women’s match data, finding that blanking may not be
optimal in the ninth end depending on the score differential. Also, the
statistical significance of hammer posession and end outcome has been
observed\citep{park2013curling}. Overall, previous research addresses
curling-specific problems that do not apply to mixed doubles curling. The
various rule changes and addition of the powerplay prompt our analysis on the
unique aspects of mixed doubles curling.

Using data from three World Championships and one Winter Olympics, this paper
offers the first analysis of strategic behavior surrounding the Power play in
professional Mixed Doubles Curling. We first show that teams tend to adopt more
defensive behaviors when using the power play, particularly in the later ends,
but that these strategic tendencies are not consistent across teams or
championships. We then focus on non-hammer team behavior, by analyzing common
counter power play shot types and find that while different opens do not
significantly affect expected scoring outcomes for the non-hammer team, they do
produce significant differences in execution quality and cumulative score
dynamics. Despite teams systemically varying their shot choice based on game
context, the data provides no evidence that any particular counter-strategy is
objectively preferable when opening an end against the Power Play.
% the key focus of the data challenge is the power play, for which we have no
% reliable strategy because this is a novel rule powerplay questions: how and
% when and why is it used how should you use it or For respond to it etc

% THESES:
% what behavior is seen during powerplay -> these behaviors are seen -> do they lead to anything

% Defensive behavior is observed during powerplay typically in later ends,
% strategic trends are not apparent across teams or championships. 

% The different types of shots to counter the powerplay do not have significant
% differences in expected HT or NHT score but they have significant differences
% in execution score & cumulative score differences. -> Teams tend to use
% different shots in different cumulative score differences, but the data
% suggests there is no reason to prefer different behavior when opening an end
% as NHT during Power play.

\section{Data}

We use a dataset from Curlit which provides 26,730 shot-by-shot stone positions
across four competitions, compiled from shot-by-shot images which are entered
into Curlit's software during the game. Included are unique IDs which
successfully distinguish individual competitions, games, ends, participants,
and preserve chronological shot order.

Perhaps the dataset's most important quality is that it is the first available
dataset for mixed doubles curling that contains stone-level coordinate data,
addressing a key limitation of previous analyses. Additionally, columns of
assessment data such as intended shot-call and execution quality are included
for every stone, providing necessary information for behavior classification.
These descriptors are based on a set of objective grading criteria, such that
that any analyst entering the data would fill each category identically.

With score results for each end, it is possible to wholly reconstruct the
entire process of each game, allowing new insights.

%some chart/table will go here, detailing the championships, number of ends, number of games,
% number of nocs, number of stones, etc etc etc

At the shot-level, the data is heavily skewed towards draws, supporting the
idea that the most dominant strategy is an opportunistic/probing strategy,
where teams continue to draw until it is determined advantageous to "break the
cluster".

This is reflected in the heatmaps of stone positions

We see that powerplay occurs in [insert Table]
% source is curlit

% Data collection was that curlit analysts entered the stone positions onto a
% softeware and evaluated shot quality on objective criteria (awarding points)

% 2 world champsionships 1 winter olympics (beijing 2022)

% This is the first data set that:
% 1. Contains stone-level coordiante data for geometric analyses
% 2. Contains powerplay-end level data for mixed doubles curling
% 3. Objective qualifiers for shot-type and shot-execution

% basic histograms which describe the data, e.g. endid, powerplay 0,1,2, count
% normal/powerplay ends how many shots every noc hits how often each shot is
% shots

\section{Methodology}
\subsection{Power Play Behavior Clustering}
% Alejandro std. coordinates estimate behavior class binary on end-level
% predictor set

% finding of mine was that behavior 

% Bridge section flowwwww

% The behavior of the HT is much easier to identify during power play
%

% Once power play behavior has been identified, clustering analysis for
% first-stone level data is used to identify ideal response types by the NHT
% [Power Play intent methodology; source: notebooks/behavior_classes.qmd]
We classify end-level intent from stone-level task data. Each shot is grouped
into offensive, defensive, or neutral based on task codes from the data
dictionary. We then split each end into phases using the Modified Free Guard
Zone: the first three shots (ShotID 7, 8, 9) are treated as the \emph{early}
phase where takeouts are prohibited, and the remaining seven shots (ShotID
16--22) are treated as the \emph{late} phase. For each team and end, we compute
early/late rates of offensive and defensive shots. A rule-based intent label is
assigned as \texttt{offensive} if late-offense $\ge$ late-defense, and
\texttt{defensive} otherwise; probing is omitted because early-phase draw-heavy
behavior is structurally forced.

To explore structure beyond the rule-based labels, we cluster end-level
patterns using standardized features: early/late offense and defense rates,
pre-end score differential, and a hammer flag. We use K-means with $k=3$ and
derive cluster intent labels based on each cluster's late-phase rates. Power
play indicators are not included in clustering to avoid leakage; power play is
compared only after labels are assigned. We also conduct sensitivity checks by
shifting the early/late boundary to include the 4th shot (older four-shot
assumption) and by testing stricter late-offense thresholds (0.55--0.65). See
\texttt{notebooks/behavior\_classes.qmd}.

\subsection{First-shot NHT Clustering}

\jy{define acronym at first and only at first occurrence.}
% Mark std. coordinates use HDBScan for clustering -> visualize w/ PCA, tSNE
% cluster quality evaluated on coverage, score, and cluster stability use welch
% ANOVA and bootstrap, use post-hoc test to determine differences between
% clusters hypothesis test: test if there is at least one group that is
% significantly different from another

% assumptions: ind, approx. normal (for anova)
The following is a brief summary of the preprocessing steps applied prior to
analysis. After the coordinates were standardized, the stone level data were
merged with the end level data and filtered to the first shot of the powerplay
right formation end. The columns for stones 3 to 6 and 9 to 12 were dropped as
they are able to be played in the first shot. Columns for stone 2 and 8 were
combined and shifted into one column for stone 2 due to both belonging to the
non-hammer team. As for stones 1 and 7, both columns represented preplaced
stones, although the designation of guard or house stone was unclear for each
row. Thus, the coordinates were swapped when either stone 1 had coordinates for
the hammer preplaced house stone or stone 7 had coordinates for the non-hammer
guard. Finally, rows with missing values were dropped. This excludes shots that
knocked out any stone before the Modified Free Guard Zone from the data.

Next, to identify patterns in stone positions, PCA, t-SNE and UMAP were applied
onto the 6-dimensional coordinate data consisting of the x and y coordinates
for stone 1, 2 and 7. HDBSCAN was selected due to non-circular clusters
appearing in the 2-D visualizations from graphing 2 components of PCA, t-SNE
and UMAP. To optimize clustering performance, Parameters of minimum cluster
size = 5 and min samples = 5 were chosen from a grid search. Cluster quality
was then assessed from DBCV score, coverage, excluded noise silhouette score
and cluster membership probabilities. Finally, the clusters were visualized by
color on 2-D components of the dimensionality reduction techniques to confirm
cluster separation.

Histograms and bar charts were used to visualize the distributions and counts
of key variables. Also, curling sheet scatter plots were used to visualize the
positions of each stone by cluster and task. Following visualization,
performance metrics across clusters were compared using Welch’s one-way ANOVA
to account for unequal group sizes and variances. Some variables were highly
skewed, which may slightly affect the p-values, but bootstrap confidence
intervals were used for the means to provide robust estimates. Where the
overall ANOVA indicated significant differences, post-hoc pairwise comparisons
were performed using the Games–Howell test. Cluster groups were defined as C0
($n$ = 27), C1 ($n$ = 73), and C2 ($n$ = 62). Values are presented as group
means with 95\% bootstrap confidence intervals to account for the skewed
distributions. Detailed results for all variables are provided in
Table~\ref{tab:cluster_test}.

To observe different countries' first non-hammer shot selection in the
powerplay, bar charts are created on both left and right powerplay formation
ends. Teams are only included if they have played more than twenty powerplay
ends. Based on the results of the clustering, similar type of shots are grouped
together. The top five countries for each group of shot type are plotted to
compare playstyles in response to the powerplay.

\section{Results}

% [Power Play intent results; source: notebooks/behavior_classes.qmd]
\subsection{Power Play Intent Results}

Across 5,274 team-end rows, the rule-based labels classify 76.8\% of ends as
offensive and 23.2\% as defensive. Power play ends show a clear defensive
shift: non-power-play ends are 80.6\% offensive (19.4\% defensive), while power
play ends are 63.6\% offensive (36.4\% defensive). The cluster-derived labels
mirror this pattern: non-power-play ends are 66.0\% offensive (34.0\%
defensive) while power play ends are 45.7\% offensive (54.3\% defensive).

The clustering yields three groups (sizes 3,239 / 1,753 / 282). One cluster is
strongly offensive (late-offense $\approx$ 0.87) and has low power play
incidence (0.17). The two defensive clusters differ in context; one shows
moderate power play incidence (0.24) and the other is heavily power-play linked
(0.80) with a positive pre-end score differential (mean $\approx$ 2.34),
consistent with a protect-a-lead context. Hammer-only shifts are smaller:
rule-based intent remains mostly offensive (76.9\% $\rightarrow$ 72.4\%), and
cluster labels tilt only slightly more defensive (51.6\% $\rightarrow$ 53.3\%
defensive).

Sensitivity checks do not change the conclusion. Using a four-shot early phase
still shows a defensive shift in power play ends (defensive 29.0\%
$\rightarrow$ 43.4\%). Stricter late-offense thresholds (0.55--0.65) yield
stable intent shares. See \texttt{notebooks/behavior\_classes.qmd} for tables
and shift summaries.

\subsection{First Shot NHT}
The HDBSCAN clustering (minimum cluster size = 5, minimum samples = 5) produced
four clusters, with a coverage rate of 0.761 and a DBCV score of 0.428. These
metrics suggest reasonable clustering structure with a fair amount of noise.
Cluster sizes ranged from 7 to 76 with 55 points labeled as noise. Cluster
assignments from HDBSCAN were probabilistic. Clusters 0–3 had mean membership
probabilities of 0.773, 0.753, 0.891, and 0.982, respectively, indicating that
Cluster 3 contained the most tightly grouped points, while Clusters 0 and 1 had
weaker assignments. Fifty-five points were labeled as noise, reflecting points
not clearly assigned to any cluster. Clusters visualized on 2-D components of
PCA, t-SNE and UMAP revealed well separated clusters with minimal overlap. With
noise excluded, silhouette score was calculated as 0.604 suggesting reasonable
to strong clustering structure.

\begin{figure}[t]
  \centering
  \includegraphics[width=1.0\textwidth]{img/fs_t-SNE.pdf}
  \caption{“t-SNE visualization of the first shot powerplay left formation
    dataset in two dimensions.
    Each point represents a sample, colored by cluster.
    Clusters indicate similarity in the feature space.”}
  \label{fig:fs_tSNE}
\end{figure}

To further characterize these clusters, we examined the spatial distribution of
stone placements and shot types for each cluster. Plotting the stone
coordinates on a curling sheet grouped by cluster reveals three clear board
positions, shown in Figure~\ref{fig:C_all}. As illustrated in
Figure~\ref{fig:C0}, the Cluster~0 pattern corresponds to situations in which
the first non-hammer team shot collides with the non-hammer guard, pushing the
stone into or near the house. Figure~\ref{fig:C1} shows that Cluster~1 is
characterized by the pre-placed stones remaining untouched, with the first shot
typically landing around the hammer team’s house stone. Similarly, Cluster~2
(Figure~\ref{fig:C2}) exhibits untouched pre-placed stones, but with the first
shot positioned in the front of the house or even higher. Cluster~3 was
identified as a distinct central draw pattern but contained a very small sample
size ($n=7$); despite strong cluster membership probabilities, it was excluded
from further analysis due to its limited representation.

\begin{figure}[t]
  \centering

  \begin{subfigure}[b]{0.32\textwidth}
    \centering
    \includegraphics[width=\textwidth]{img/C0.pdf}
    \caption{Cluster 0: Tick shot}
    \label{fig:C0}
  \end{subfigure}
  \hfill
  \begin{subfigure}[b]{0.32\textwidth}
    \centering
    \includegraphics[width=\textwidth]{img/C1.pdf}
    \caption{Cluster 1: Draw}
    \label{fig:C1}
  \end{subfigure}
  \hfill
  \begin{subfigure}[b]{0.32\textwidth}
    \centering
    \includegraphics[width=\textwidth]{img/C2.pdf}
    \caption{Cluster 2: Front/Guard}
    \label{fig:C2}
  \end{subfigure}

  \caption{Coordinate visualizations after the first shot in the powerplay left formation.
    Each panel highlights a different cluster, which can be identified as shot types:
    tick (Panel A), draw around the guard (Panel B), and front/guard (Panel C)}
  \label{fig:C_all}
\end{figure}

We next examined the distribution of shot types within each cluster. Histograms
of proportion of each task grouped by cluster conclude that Cluster 0 having a
roughly equal proportion of raise/tapbacks and wick/soft-peeling shots. Cluster
1 only consists of draws, and Cluster 2 has an equal number of fronts and
guards.

Finally, we assessed whether these cluster groupings corresponded to
differences in game outcomes. Cumulative score difference differed
significantly across clusters (Welch’s ANOVA, $p=0.013$;
Table~\ref{tab:cluster_test}), with the highest mean observed in C1 (6.16, 95\%
CI 5.26--7.08) and the lowest in C0 (3.44, 95\% CI 1.89--5.04). Post-hoc
Games–Howell tests indicated a significant difference between C0 and C1
($p=0.016$), while differences between C0 and C2 ($p=0.343$) and C1 and C2
($p=0.205$) were not statistically significant. Points also varied across
clusters (Welch’s ANOVA, $p=0.005$; Table~\ref{tab:cluster_test}), with a
significant post-hoc difference between C1 and C2 ($p=0.004$), but not between
C0 and C1 ($p=0.326$) or C0 and C2 ($p=0.845$). Hammer and non-hammer scores
did not differ significantly across clusters (Welch’s ANOVA, $p=0.769$ and
$p=0.833$, respectively; Table~\ref{tab:cluster_test}). Overall, these results
indicate that cumulative score difference and points showed cluster-specific
variation, while hammer-related outcomes were similar across groups.

\jy{n is not $n$}
\begin{table}[t]
  \centering
  \includegraphics[width=0.8\textwidth]{img/cluster_test.pdf}
  \caption{Group means (with 95\% bootstrap confidence intervals) for each
    outcome variable by HDBSCAN cluster. p-values are from Welch’s one-way
    ANOVA. Significant pairwise differences were assessed using Games–Howell
    post hoc tests.}
  \label{tab:cluster_test}
\end{table}

\section{Discussion}

% 1st - summarize 2nd - why is your approach the best 3rd - limitations. what
% question might someone else want to answer that we can't address

% bring up the most interesting and insightful results bring up the
% implications, make actionable recommendations

% alejandro: curling strategy says to enter with a game-plan game-plans for how
% to use powerplay are not observed with this data. 

% [Power Play intent findings; source: notebooks/behavior.qmd]
\subsection{Power Play Intent Findings}

The core finding is that power play ends are \emph{more defensive} in the late
phase, despite the rule's intent to open the sheet. The early phase is
overwhelmingly offensive for both labels, which is consistent with the MFGZ
constraint; the late phase is where intent diverges sharply. The defensive
shift is not driven entirely by hammer-team behavior, which only shifts
modestly, suggesting that non-hammer responses or end-level context are key
drivers.

Overall, the evidence supports the analysts' claim that power play does not
simply increase offensive play and may even induce defensive behavior. The
mechanism is likely contextual (score state, opponent response, or end timing),
which motivates the planned hypothesis checks on score differential, end
outcomes, and context dependence. See \texttt{notebooks/behavior.qmd} for the
full interpretation block.

\subsection{First Shot NHT}
Analysts have observed different responses for countering the powerplay,
although there has been no studies to identify exactly which shots teams hit
and their effect on performance and context metrics. By clustering the board
positions after the non-hammer team hits their first shot, we can observe the
three distinct patterns in curling mixed doubles championships. The tick, guard
and draw around the guard are all common shots that occur in response to the
powerplay.

Welch ANOVA was used to compare differences between clusters, revealing
significant differences in cumulative score difference and points in the
non-missed shots and no significant differences in hammer or non-hammer score.
Post Hoc Games-Howell tests show that significant cumulative score differences
between the tick and draw around the guard. All shots are utilized when the
hammer team has the score advantage. However, the tick shot is thrown at when
the score difference is lesser, while the draw is thrown when the hammer team
has a blowout lead.

Teams may have tendencies to throw less risky shots when they are behind less
in the score compared to when they are down many points. Nevertheless, as there
are no significant differences in hammer and non-hammer score between the
clusters, shot choice may not have an important role in end score performance.
We suggest that when countering the powerplay as the non-hammer team, that
teams choose the starting player based on their overall skill in executing one
of the three options observed.

When including all of the powerplay ends, certain countries play the draw,
guard or tick shot a lot more than others. The bar chart of proportion of task
groups by nation indicate the clear difference in strategy in throwing the
first stone during powerplay ends (Figure~\ref{fig:noc_pptask}). The tasks are
grouped by the results discovered through clustering. All teams tend to hit
more draws and tick shots, with Korea hitting 67\% of their shots as draws.
Germany plays the most guards at 48\%. Sweden along with Scotland hit a large
majority of their shots as Raise/Tapbacks and Wick/Soft-Peels. These
differences confirm the analysts' intutition about varying responses to
countering the powerplay.

\begin{figure}[t]
  \centering
  \includegraphics[width=\textwidth]{img/noc_pptask.pdf}
  \caption{“Proportions of grouped tasks hit by country. Data includes all powerplay
    ends and is limited to the first stone. Only the top five teams with more than 20
    powerplay ends are shown.”}
  \label{fig:noc_pptask}
\end{figure}

\subsection{Limitations}
\jy{Could be brief, because this is not exactly an academic paper but a
  report for reference for the coaching team.}

% lack of powerplay sample size very skewed data. task, points, expected hammer
% points data quality. i.e. inconsistent coordinate assignments, incosistent
% column assignments

% freeze vs draw inconsistency 

% imbalanced data with logical end devisions before vs after MFGZ 

% one of the clusters does not identify raise / tap-backs on the hammer stone

% the draw cluster has a very small sample size but is recognized as a cluster

Some limitations included small powerplay sample size and heavily skewed data
in multiple columns such as task, points, hammer score and points. Also,
coordinates and columns were sometimes assigned inconsistently. For example, a
draw may be identified as a freeze.

\subsection{Conclusion}
% quick summary

%\section{Appendices} if there is a need for code appendix or otherwise, we'll
%put it at the botton.

\jy{Do you have team strategy profiles?}
\bibliographystyle{apalike}
\bibliography{references}
%i have my latex in vscode compiled to use natbib, not bibtex. it will work for
% you but i havent got mine working yet, once i set it back we will use APA

\end{document}
