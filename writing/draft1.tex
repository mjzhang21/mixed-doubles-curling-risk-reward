\documentclass{article}
\usepackage{amsmath}
\usepackage{fullpage}
\usepackage{graphicx}
\usepackage[utf8]{inputenc}
\usepackage{natbib}
\usepackage{xcolor}

\title{Placeholder Title\\
  \large [Subtitle (optional)]}
\author{[Mark Zhang]\thanks{Department of Statistics, University of Connecticut.} \and
  [Alejandro Haerter]\thanks{Department of Statistics, University of Connecticut.}}
\date{[15 January 2026]}



\begin{document}

\begin{titlepage}
  \maketitle
  % i may add a table of contents here.
\end{titlepage}

% Note: % is comment character, \ backslash is the escape character (needed for $, %, etc.)

% New paragraphs will be formed automatically similar to quarto, just leave a blank line in between
% lines. Inline math can be called by wrapping with $, for example $ math equation$.

\section{Abstract}

\section{Introduction}

\section{Data}


\section{Methods}


\section{Simulation}


\section{Discussion}


\section{Conclusion}


%\section{Appendices}
%if there is a need for code appendix or otherwise, we'll put it at the botton. 

\bibliography{../references.bib}
%i have my latex in vscode compiled to use natbib, not bibtex. it will work for you
% but i havent got mine working yet, once i set it back we will use APA
\end{document}
