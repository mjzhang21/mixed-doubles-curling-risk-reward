\documentclass{article}
\usepackage{amsmath}
\usepackage[margin=1in]{geometry}
\usepackage{graphicx}
\usepackage[utf8]{inputenc}
\usepackage{natbib}
\usepackage{xcolor}
\usepackage{caption}

\title{Analysis of Power Play Behavior and Counter Play in Mixed Doubles Curling\\
  \large CSAS 2026 Data Challenge Submission}

\date{[15 January 2026]}



\begin{document}

\begin{titlepage}
  \maketitle
  % i may add a table of contents here.
\end{titlepage}

% Note: % is comment character, \ backslash is the escape character (needed for $, %, etc.)

% New paragraphs will be formed automatically similar to quarto, just leave a
% blank line in between lines. Inline math can be called by wrapping with $, for
% example $ math equation$.

\section{Introduction}

% briefly outline what the purpose of the data challenge is exactly -> there isn
% o data anlysis done on mixed doubles curling all mixed doubles curling
% strategy and intuition relies on strategy formulated from the tradtional game.
% this is why we have the data challenge, is so that we can develop strategy
% independent of the 4 person game.

% the key focus of the data challenge is the power play, for which we have no
% reliable strategy because this is a novel rule powerplay questions: how and
% when and why is it used how should you use it or For respond to it etc

% THESES:
% what behavior is seen during powerplay -> these behaviors are seen -> do they lead to anything

% Defensive behavior is observed during powerplay typically in later ends,
% strategic trends are not apparent across teams or championships. 

% The different types of shots to counter the powerplay do not have significant
% differences in expected HT or NHT score but they have significant differences
% in execution score & cumulative score differences. -> Teams tend to use
% different shots in different cumulative score differences, but the data
% suggests there is no reason to prefer different behavior when opening an end
% as NHT during Power play.

\subsection{Literature Review}

% bring up scarcity of resources and limitations of extant literature

\section{Data}

% source is curlit

% Data collection was that curlit analysts entered the stone positions onto a
% softeware and evaluated shot quality on objective criteria (awarding points)

% 2 world champsionships 1 winter olympics (beijing 2022)

% This is the first data set that:
% 1. Contains stone-level coordiante data for geometric analyses
% 2. Contains powerplay-end level data for mixed doubles curling
% 3. Objective qualifiers for shot-type and shot-execution

% basic histograms which describe the data, e.g. endid, powerplay 0,1,2, count
% normal/powerplay ends how many shots every noc hits how often each shot is
% shots

\section{Methodology}
\subsection{Power Play Behavior Clustering}
% Alejandro std. coordinates estimate behavior class binary on end-level
% predictor set

% finding of mine was that behavior 

% Bridge section flowwwww

% The behavior of the HT is much easier to identify during power play
%

% Once power play behavior has been identified, clustering analysis for
% first-stone level data is used to identify ideal response types by the NHT

\subsection{First-shot NHT Clustering}

% Mark std. coordinates use HDBScan for clustering -> visualize w/ PCA, tSNE
% cluster quality evaluated on coverage, score, and cluster stability use welch
% ANOVA and bootstrap, use post-hoc test to determine differences between
% clusters hypothesis test: test if there is at least one group that is
% significantly different from another

% assumptions: ind, approx. normal (for anova)
The following is a brief summary of the preprocessing steps applied prior to
analysis. After the coordinates were standardized, the stone level data were
merged with the end level data and filtered to the first shot of the end. The
columns for stones 3 to 6 and 9 to 12 were dropped as they are able to be
played in the first shot. Columns for stone 2 and 8 were combined and shifted
into one column for stone 2 due to both belonging to the non-hammer team. As
for stones 1 and 7, both columns represented preplaced stones, although the
designation of guard or house stone was unclear for each row. Thus, the
coordinates were swapped when either stone 1 had coordinates for the hammer
preplaced house stone or stone 7 had coordinates for the non-hammer guard.
Finally, rows with missing values were dropped. This excludes shots that
knocked out any stone before the Modified Free Guard Zone from the data.

Next, to identify patterns in stone positions, PCA, t-SNE and UMAP were applied
onto the 6-dimensional coordinate data consisting of the x and y coordinates
for stone 1, 2 and 7. HDBSCAN was selected due to non-circular clusters
appearing in the 2-D visualizations from graphing 2 components of PCA, t-SNE
and UMAP. To optimize clustering performance, Parameters of minimum cluster
size = 5 and min samples = 5 were chosen from a grid search Cluster quality was
then assessed from DBCV score, coverage, excluded noise sillouhete score and
cluster membership probabilities. Finally, the clusters were visualized by
color on 2-D components of the dimensionality reduction techniques to confirm
cluster separation.

Histograms and bar charts were used to visualize the distributions and counts
of key variables. Also, curling sheet scatterplots were used to visualize the
positions of each stone by cluster and task.

Following visualization, performance metrics across clusters were compared
using Welch’s one-way ANOVA to account for unequal group sizes and variances.
Post-hoc pairwise comparisons were performed using the Games–Howell test where
the overall ANOVA indicated significant differences. Cluster groups were
defined as C0 (n=27), C1 (n=73), and C2 (n=62). Values are presented as group
means with 95\% bootstrap confidence intervals, and detailed results for all
variables are provided in Table~\ref{tab:cluster_test}.

\begin{table}[t]
  \centering
  \includegraphics[width=0.8\textwidth]{img/cluster_test.pdf}
  \caption{Group means (with 95\% bootstrap confidence intervals) for each
    outcome variable by HDBSCAN cluster. p-values are from Welch’s one-way
    ANOVA. Significant pairwise differences were assessed using Games–Howell
    post hoc tests.}
  \label{tab:cluster_test}
\end{table}

\section{Results}

The HDBSCAN clustering (minimum cluster size = 5, minimum samples = 5) produced
four clusters, with a coverage rate of 0.761 and a DBCV score of 0.428. These
metrics suggest reasonable clustering structure with a fair amount of noise.
Cluster sizes ranged from 7 to 76 with 55 points labeled as noise. Cluster
assignments from HDBSCAN were probabilistic. Clusters 0–3 had mean membership
probabilities of 0.773, 0.753, 0.891, and 0.982, respectively, indicating that
Cluster 3 contained the most tightly grouped points, while Clusters 0 and 1 had
weaker assignments. Fifty-five points were labeled as noise, reflecting points
not clearly assigned to any cluster. Clusters visualized on 2-D components of
PCA, t-SNE and UMAP revealed well separated clustrs with minimal overlap. With
noise excluded, sihlouette score was calculated as 0.604 suggesting reasonable
to strong clustering structure.

To further characterize these clusters, we examined the spatial distribution of
stone placements and shot types for each cluster. Plotting the stone
coordinates on a curling sheet grouped by cluster reveals three clear board
positions which have been identified. The Cluster 0 plot illustrates the first
non-hammer team shot collides with the non-hammer guard pushing into around the
house or near the button. The Cluster 1 plot has preplaced stones untouched,
with the first shot landing around the hammer house stone. Cluster 2 also shows
the preplaced stones untouched and the first shot in the front of the house or
even higher. Cluster 3 had a very small sample size but is revealed to be a
central draw. Although identified as a distinct cluster with strong membership
probability, Cluster 3 (n=7) was excluded from further analyses due to its
small size.

We next examined the distribution of shot types within each cluster. Histograms
of proportion of each task grouped by cluster conclude that Cluster 0 having a
roughly equal distribution of raise/tapbacks and wick/soft-peeling shots.
Cluster 1 only consists of draws and Cluster 2 consists of fronts and guards.

Finally, we assessed whether these cluster groupings corresponded to
differences in game outcomes. Cumulative score difference differed
significantly across clusters (Welch’s ANOVA, $p=0.013$;
Table~\ref{tab:cluster_test}), with the highest mean observed in C1 (6.16, 95\%
CI 5.26--7.08) and the lowest in C0 (3.44, 95\% CI 1.89--5.04). Post-hoc
Games–Howell tests indicated a significant difference between C0 and C1
($p=0.016$), while differences between C0 and C2 ($p=0.343$) and C1 and C2
($p=0.205$) were not statistically significant. Points also varied across
clusters (Welch’s ANOVA, $p=0.005$; Table~\ref{tab:cluster_test}), with a
significant post-hoc difference between C1 and C2 ($p=0.004$), but not between
C0 and C1 ($p=0.326$) or C0 and C2 ($p=0.845$). Hammer and non-hammer scores
did not differ significantly across clusters (Welch’s ANOVA, $p=0.769$ and
$p=0.833$, respectively; Table~\ref{tab:cluster_test}). Overall, these results
indicate that cumulative score difference and points showed cluster-specific
variation, while hammer-related outcomes were similar across groups.

\section{Discussion}

% 1st - summarize 2nd - why is your approach the best 3rd - limitations. what
% question might someone else want to answer that we can't address

% bring up the most interesting and insightful results bring up the
% implications, make actionable recommendations

% alejandro: curling strategy says to enter with a game-plan game-plans for how
% to use powerplay are not observed with this data. 

\subsection{Limitations}
% lack of powerplay sample size very skewed data. task, points, expected hammer
% points data quality. i.e. inconsistent coordinate assignments, incosistent
% column assignments

% freeze vs draw inconsistency 

% imbalanced data with logical end devisions before vs after MFGZ 

% one of the clusters does not identify raise / tap-backs on the hammer stone

% the draw cluster has a very small sample size but is recognized as a cluster

\subsection{Conclusion}
% quick summary

%\section{Appendices} if there is a need for code appendix or otherwise, we'll
%put it at the botton. 
\bibliographystyle{plainnat}
\bibliography{../references}
%i have my latex in vscode compiled to use natbib, not bibtex. it will work for
% you but i havent got mine working yet, once i set it back we will use APA

\end{document}

