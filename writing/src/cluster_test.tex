\documentclass{article}

\usepackage[utf8]{inputenc} % only needed for pdflatex
\usepackage{booktabs, tabularx, threeparttable, graphicx, caption}


\begin{document}

% Optional: reduce column padding and font size for compact table
\setlength{\tabcolsep}{4pt}
\small

\begin{table}[h!]
\centering
\caption{Values are presented as mean (95\% CI) or median (Q1, Q3). 
C0: Cluster 0, C1: Cluster 1, C3: Cluster 3.
P values indicate the statistical inference result of overall comparisons.}
\begin{threeparttable}
\begin{tabularx}{\textwidth}{>{\raggedright\arraybackslash}X c c c c}
\toprule
Variables & C0 (n=35) & C1 (n=75) & C3 (n=67) & p-value \\
\midrule
Cumulative score difference\tnote{*}    & 3.51 (2.06--4.97) & 6.00 (5.07--6.93) & 4.84 (3.73--5.94) & 0.015 \\
Points\tnote{\textdagger}               & 4.00 (3.00--4.00) & 3.50 (3.00--4.00) & 4.00 (3.00--4.00) & 0.003 \\
Hammer score\tnote{\textdagger}         & 1.00 (0.50--2.00) & 1.00 (1.00--2.00) & 1.00 (0.50--2.00) & 0.958 \\
Non-hammer score\tnote{\textdagger}     & 0.00 (0.00--0.50) & 0.00 (0.00--0.00) & 0.00 (0.00--0.00) & 0.904 \\
\bottomrule
\end{tabularx}

\begin{tablenotes}[flushleft]
\small
\item[*] Welch’s one-way ANOVA
\item[\textdagger] Kruskal-Wallis H test
\end{tablenotes}
\end{threeparttable}
\end{table}

\end{document}
